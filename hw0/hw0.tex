\documentclass[../main]{subfiles}

\begin{document}

\ifSubfilesClassLoaded{%
    \title{Categories Warm-Up}
    \subtitle{}
    \author{}
    \contact{Summer 2024}
    \date{NeuPPL Category Seminar}}{%
    \title{Categories Warm-Up}
    \subtitle{}
    \author{}
    \contact{}
    \date{}}

\makehmtitle%

\section*{Preorders as Categories}

Let \(X\) be a set. A \textbf{preorder} on \(X\) is a reflexive and transitive
relation \(\_\leq\_ \subseteq X \times X\). For example, any partial order is a
preorder (but not all preorders are partial orders!). We will show that
preordered sets form a special class of categories.

\begin{exercise}
  Let \(\pwrap{X,\leq}\) be a preordered set. Define a category
  \(\Cl\pwrap{X,\leq}\) in the following manner.
  \begin{itemize}
  \item The objects of \(\Cl\pwrap{X,\leq}\) are the elements of \(X\).
  \item There is precisely one arrow \(x \to y\) in \(\Cl\pwrap{X,\leq}\) in the
    case that \(x \leq y\).
  \end{itemize}
  Verify that \(\Cl\pwrap{X,\leq}\) is a category. Can any of the hypotheses on
  the relation \(\leq\) be relaxed while still ensuring that
  \(\Cl\pwrap{X,\leq}\) is a category?

  The category \(\Cl\pwrap{X,\leq}\) is called the \textbf{classifying category}
  of the preorder \(\pwrap{X,\leq}\).
\end{exercise}

\begin{exercise}
  \begin{itemize}
  \item Unpack what it means for two objects \(x,y\) of \(\Cl\pwrap{X,\leq}\) to
    be isomorphic, in terms of the preordered set \(\pwrap{X,\leq}\).
  \item Unpack what is a functor \(\Cl\pwrap{X,\leq} \to \Cl\pwrap{Y,\leq}\), in
    terms of a function between preordered sets \(\pwrap{X,\leq} \to
    \pwrap{Y,\leq}\).
  \item The opposite category \(\Cl\pwrap{X,\leq}^\op\) is the classifying
    category of another preordered set \(\pwrap{X',\leq}\). What is this
    preordered set?
  \end{itemize}
\end{exercise}

\begin{exercise}
  Let \(\Omega \eqdef \bwrap{\False, \True}\) with the usual preordering
  \(\False \leq \True\). To keep things clean, let's also write \(\Omega\) for
  \(\Cl\pwrap{\Omega,\leq}\). Given an object \(x\) of \(\Cl\pwrap{X,\leq}\), we
  can write a functor \(\downarrow x : \Cl\pwrap{X,\leq}^\op \to \Omega)\):
  \[%
    \downarrow x (y) \eqdef
    \begin{cases}
      \True & \text{if \(y \leq x\)} \\
      \False & \text{otherwise}.
    \end{cases}
  \]%
  Write \(\downarrow x \stackrel{\text{nat}}{\leq} \downarrow y\) in the
  case that, for any object \(z\) of \(\Cl\pwrap{X,\leq}\),
  \(\downarrow x(z) \leq \downarrow y(z)\). Show for any pair of elements
  \(x,y \in X\),
  \[%
    x \leq y \text{ if and only if } \downarrow x
    \stackrel{\text{nat}}{\leq} \downarrow y.
  \]%
\end{exercise}

\section*{Monads for Preorders (Extra)}

Given two categories \(C,D\) and a functor \(F: C \to D\), a \textbf{right
  adjoint} to \(F\) is a functor \(G : D \to C\) satisfying the following
universal property.
\[%
  \begin{array}{c}
    f : F X \to Y \\ \hline\hline
    f^\dagger : X \to G Y.
  \end{array}
\]%
We read this as: \emph{for any construction of an arrow from \(F X\) to \(Y\) in
  \(D\), there exists a unique dual construction of an arrow form \(X\) to \(G
  Y\) in \(C\).}
In the case where \(C\) and \(D\) are the classifying categories of preordered
sets, \(C = \Cl\pwrap{X,\leq}\) and \(D = \Cl\pwrap{Y,\leq}\), then there is
only one way to make an arrow from \(FX\) to \(Y\) and from \(X\) to
\(GY\). Then the above universal property reduces to
\[%
  \begin{array}{c}
    FX \leq Y\\ \hline\hline
    X \leq GY
  \end{array}
\]%
This pair \(F \dashv G\) is called a \textbf{Galois connection}.

\begin{exercise}
  Given a Galois connection \(F \dashv G\), where \(F : \Cl\pwrap{X,\leq} \to
  \Cl\pwrap{Y,\leq}\) and \(G: \Cl\pwrap{Y,\leq} \to \Cl\pwrap{X,\leq}\), we can
  apply \(G\) after \(F\) to obtain a ``loop'':
  \[%
    G F : \Cl\pwrap{X,\leq} \to \Cl\pwrap{X,\leq}.
  \]%
  Show that \(GF\) is a \textbf{closure operator}, i.e., for any object \(x\) of
  \(\Cl\pwrap{X,\leq}\), \(x \leq GF x\) and \(GF \pwrap{ GF x } \leq GF x\).
\end{exercise}

\begin{exercise}
  We can reverse the previous exercise. Apply \(F\) after \(G\) to obtain
  another ``loop'':
  \[%
    FG : \Cl\pwrap{Y,\leq} \to \Cl\pwrap{Y,\leq}.
  \]%
  Show that \(FG\) is an \textbf{interior operator}, i.e., for any object \(y\)
  of \(\Cl\pwrap{Y,\leq}\), \(FG y \leq y\) and \(FG y \leq FG \pwrap{FG y}\).
\end{exercise}

  Here is an example of a closure operator which will appear in different forms
later. Let \(\pwrap{X,\leq}\) be a preorder, as before. We saw in
Exercise~\ref{ex:3} that we can ``embed'' the elements of \(X\) into maps
\(\Cl\pwrap{X,\leq}^\op \to \Omega\). In fact, we can make the set of all such
maps into a preorder using \(\stackrel{\text{nat}}{\leq}\). Given a pair of maps
\(F,F' : \Cl\pwrap{X,\leq}^\op \to \Omega\), write \(F
\stackrel{\text{nat}}{\leq} F'\) in the case that, for any element \(x \in X\),
we have
\[%
  F x \leq F x '.
\]%
A \textbf{Lawvere-Tierney topology} on \(\pwrap{X,\leq}\) is a closure
operator \(J\) on the set of maps \(\Cl\pwrap{X,\leq}^\op \to \Omega\)
satisfying:
\begin{itemize}
\item \(J\) is \textbf{idempotent}: \(J(J F) \natsubseteq J (F)\) (and \(J(F)
  \natsubseteq J(J(F))\).
\item \(J\) preserves infima.
\end{itemize}

\end{document}
