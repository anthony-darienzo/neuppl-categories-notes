\documentclass{../thesis-note}

\title{Presheaves and Indexed Set Theory}
\author{NeuPPL Category Seminar}
\date{Summer 2024}

\definecolor{NortheasternRed}{RGB}{200,16,46}

\titlecolor{NortheasternRed}

\newcommand\Prop[1]{\text{{\normalfont{Prop}}}{\pwrap{{#1}}}}
\newcommand\Dwd{\text{\normalfont Dwd}}
\newcommand\natleq{\mathrel{\stackrel{\text{nat}}{\leq}}}
\DeclareMathOperator\dset{\downarrow}
\DeclareMathOperator\Mod{\normalfont Mod}
\DeclareMathOperator\el{el}
\newcommand\Pcal{{\mathcal{P}}}
\newcommand\iSet[1]{{{#1}\text{\normalfont{-Set}}}}

\tikzset{snake it/.style={decorate, decoration=snake}}

\usepackage{bussproofs}

\begin{document}

\maketitle%

The first half of these notes developed a categorical perspective on the theory
of preorders. One key step in this theory is the observation that a preorder
\(\pwrap{X,\leq}\) may be treated as a family of indexed propositions
\(\Cl\pwrap{X,\leq}^\op \to \Omega\). This allowed us to lift operations on
truth values \(\Omega\) to universal constructions in a preorder. In a category
\(C\), we can no longer treat objects like indexed \emph{propositions}, but
there is a way to generalize this philosophy. Instead, we need to study indexed
sets, i.e., \emph{presheaves}. Similar to the preorder situation, operations
between sets can be lifted to universal constructions in a category.

\section{A brief account of categories}

A category is a collection of objects and arrows between them which can be
composed. There are myriad situations where such structure arises. Perhaps two
archetypes are interfaces and theories. Qua interface, a category is a
collection of states, called \emph{objects}, and families of transitions between
states, called \emph{arrows}. Functors \(F : C \to \Set\) can be seen as
concrete implementations of this interface. This philosophy is explained well
in~\cite{Fong2019}. This approach also forms the semantics for algebraic
effects. For example, algebras of the \texttt{Writer m} monad, for some monoid
\texttt{m} are described by the category
\begin{figure}[h]
  \centering
  \begin{tikzpicture}[node distance=1.75cm]
    \tikzstyle{place}=[circle,thick,draw=gray!75,fill=gray!20,minimum size=6mm]
    \begin{scope}
      \node [place] (obj) {};
      \draw[thick,->, shorten >=1pt] (obj) to [out=90,in=135,loop,looseness=4.8]
      (obj);
      \draw[thick,->, shorten >=1pt] (obj) to [out=155,in=200,loop,looseness=4.8]
      (obj);
      \draw[thick,->, shorten >=1pt] (obj) to [out=220,in=265,loop,looseness=4.8]
      (obj);
      \node[rotate=45] (dots) at ($ (obj) + (-30:0.5) $) {$\cdots$};
      \draw[thick,->, shorten >=1pt] (obj) to
      [out=375,in=420,loop,looseness=4.8] (obj);
    \end{scope}
  \end{tikzpicture}
\end{figure}
consisting of a single object and an arrow for every element of the monoid
\texttt{m}. Functors from this category to, e.g., \(\Set\) identifies a set
\(X\) (the image of the single object) and a family of functions \(X \to X\) for
every element of \texttt{m}. This is a \texttt{Writer m} algebra.

Qua theory, a category \(C\) is the classifying category of
\emph{some theory}. Functors out of this category are the same as models of this
theory. Given a propositional theory \(T\), we saw that functors from \(\Cl(T)\)
to a preorder \(X\) were the same as models of \(T\) in \(X\). This perspective
is not fundamentally different to the interface perspective.

We will be using categories to describe the logical relations of a programming
language with state, e.g., name allocation or step\-/indexing. We will identify
a category \(C\) whose objects will represent states of a heap. Arrows of \(C\)
will identify how a heap state may be extended with fresh names. We will write
our logical relations as sets indexed by heap states, i.e., functors out of
\(C\) into \(\Set\).

For the sake of completeness, here is a definition of a (small) category.
\begin{definition}
  A \textbf{small category} \(C\) is a pair of sets \(\pwrap{C^{(0)},
    C^{(1)}}\), called its set of \textbf{objects} and its set of
  \textbf{arrows}, respectively, and a family of functions.
  \begin{itemize}
  \item A pair \(s,t : C^{(1)} \to C^{(0)}\) of \textbf{source} and
    \textbf{target} functions. For any arrow \(f \in C^{(1)}\), \(s(f)\) is its
    source object, and \(t(f)\) is its target object. We usually write this as
    \(f : s(f) \to t(f)\).
  \item A function \(1: C^{(0)} \to C^{(1)}\) sending every object \(X \in
    C^{(0)}\) to its \textbf{identity arrow} \(1_X : X \to X\).
  \item A partial function \(C^{(1)} \times C^{(1)} \to C^{(1)}\) called
    \textbf{composition}. \(m(g,f)\) is defined when \(s(g) = t(f)\).
  \end{itemize}
  These functions are required to satisfy two axioms. First, \(m\) must be an
  associative operation:
  \[%
    m(h,m(g,f)) = m(m(h,g),f),
  \]%
  wherever defined. Second, \(1\) must be a left and right identity for \(m\):
  \[%
    m(f,1_{s(f)}) = m(1_{t(f)}, f) = f,
  \]%
  for any \(f \in C^{(1)}\).
\end{definition}
Usually composition \(m(g,f)\) in a category is written as \(g\circ f\) or by
juxtaposition \(gf\).
\begin{definition}
  Given small categories \(C\) and \(D\), a \textbf{functor} \(F: C \to D\) is a
  family of functions \(F^{(0)} : C^{(0)} \to D^{(0)}\) and \(F^{(1)} : C^{(1)}
  \to D^{(1)}\) which preserve the functions \(s,t,1,\) and \(m\).
\end{definition}
Given a category \(C\), swapping the functions \(s\) and \(t\) yields a new
category, \(C^\op\), called its \textbf{opposite category}. This is a
generalization of the relation between \(\leq\) and \(\geq\) for preordered
sets.

There is one more perspective on categories which ties this back to the study on
preorders. Recall, for a preordered set \(\pwrap{X,\leq}\), the classifying
category \(\Cl\pwrap{X,\leq}\) is \emph{thin}---between any two objects \(x,y\)
in \(\Cl\pwrap{X,\leq}\), there is at most one arrow \(x \to y\). A category
\(C\) is an intensional form of a preorder. For a category, it is not enough to
know that there exists an arrow between objects \(X\) and \(Y\), the specific
choice of arrow \(f: X \to Y\) is now relevant. Similarly, \(\Set\) is the
intesional analogue of truth values: it is not enough to know that a set \(X\)
is nonempty; rather, the specific choice of element \(x \in X\) matters. This
perspective tells us that universal constructions ofr categories are
intensional/proof\-/relevant variants of the universal constructions for
preorders. We saw some of these generalizations in the notes on Galois
connections.

\section{Indexed sets}

We begin by describing a presheaf as an indexed set. By indexed set, we mean
indexed in the following manner. Given a family of sets \(\bwrap{ X_\alpha \mid
  \alpha \in I }\), we can describe this family with a function
\[%
  \pi : X \to I,
\]%
where \(X \eqdef \coprod_{\alpha \in I} X_\alpha\). Given a function \(x \in
X\), \(x\) is contained in some \(X_\alpha\) for a unique \(\alpha\). \(\pi(x)\)
is defined to be this index \(\alpha\). In the case where \(I\) is replaced with
a category \(C\), we follow the same idea, but we introduce some extra structure
to make the set \(X\) compatible with the arrows of \(C\).
\begin{definition}
  Let \(C\) be a small category. A \textbf{\(C\)\-/indexed set} is a set \(X\)
  equipped with a function \(\pi : X \to C^{(0)}\) and a partial function
  \(\mu : X \times C^{(1)} \to X\) satisfying the following constraints.
  \begin{itemize}
  \item \(\mu(x,f)\) is defined whenever \(t(f) = \pi(x)\).
  \item \(\pi\pwrap{\mu(x,f)} = s(f)\).
  \item \(\mu(x,1_{\pi(x)}) = x\).
  \item If \(g \circ f\) is defined, then \(\mu(x,g\circ f) =
    \mu(\mu(x,g),f)\), wherever defined.
  \end{itemize}
  Anticipating an equivalence between \(C\)\-/indexed sets and presheaves on
  \(C\), we could write \(\restr{x}{f}\) in lieu of \(\mu(x,f)\).
\end{definition}
Calling this a \(C\)\-/indexed set is admiteddly uncommon. Another name for a
\(C\)\-/indexed set is a \emph{discrete opfibration}. This is more precise, and
it alludes to a more general notion, called a \emph{fibration of categories}. On
the other hand, there is a hint of the notion of \texttt{Writer m} algebra in a
\(C\)\-/indexed set. A category with one object, i.e., a category \(C\) where
\(C^{(0)} \cong \bwrap{\ast}\), is the same structure as a monoid on the set
\(C^{(1)}\). In this sense categories are generalized monoids, where the
multiplication is a partial operation. Given that categories are generalized
monoids, \(C\)\-/indexed sets are generalized modules.
\begin{proposition}
  Let \(C\) be a category with one object. Then a \(C\)\-/indexed set
  \(\pwrap{X,\pi,\mu}\) is the same data as a right\-/action of the monoid
  \(C^{(1)}\) on the set \(X\).
\end{proposition}
\begin{proof}
  Consider a \(C\)\-/indexed set \(\pwrap{X,\pi,\mu}\). Since \(C^{(0)} \cong
  \bwrap{\ast}\), the projection \(\pi : X \to C^{(0)}\) can only be the map
  sending any element of \(X\) to the single element of \(C^{(0)}\). In this
  case, \(\mu : X \times C^{(1)} \to X\) is total. We define the right action on
  \(X\) by \(C^{(1)}\) by the rule
  \[%
    x \cdot f \eqdef \mu(x,f).
  \]%
  Conversely, given a right action of \(C^{(1)}\) on \(X\), we give \(X\) the
  structure of a \(C\)\-/indexed set in the following manner. The projection
  \(\pi\) is unique, since \(C^{(0)}\) is a singleton set. This leaves the map
  \(\mu : X \times C^{(1)} \to X\), which we define using the same rule as
  before.
  \[%
    \mu(x,f) \eqdef x \cdot f.
  \]%
\end{proof}
Due to the above proposition, presheaves are occaisionally called \emph{right
  \(C\)\-/modules}. Section V.7 of \cite{MacLane1994} describes the theory
of right \(C\)\-/modules. Furthermore, to feed mathematicians' preference for
infix notation, we will now write \(x \cdot f\) instead of \(\mu(x,f)\) for any
\(C\)\-/indexed set.

Given a \(C\)\-/indexed set \(\pwrap{X,\pi,\cdot}\), we can lift the categorical
structure on \(C\) to \(X\). In order to give this category a name, we will
anticipate presheaves and abuse notation slightly to call this category
\(\el(X)\), the \emph{category of elements} of \(X\).
\begin{definition}
  Given the \(C\)\-/indexed set \(\pwrap{X,\pi,\cdot}\), its \textbf{category of
    elements} is the following category \(\el(X)\). The objects of \(\el(X)\)
  are the elements of \(X\). Furthermore, there is an arrow
  \(\widetilde{f} : x \to y\) in \(\el(X)\) for every arrow \(f \in C^{(1)}\)
  such that \(x = y \cdot f\). Composition of arrows in \(\el(X)\) is the same
  as composition in \(C\).
\end{definition}
It can be helpful to have a picture for how the arrows in \(\el(X)\) are related
to arrows in \(C\).
\[%
  \begin{tikzcd}
    x \arrow[r,dashed,"\widetilde{f}"] \arrow[d,snake it] & y \arrow[d,snake it] \\
    \pi(x) \arrow[r,"f"] & \pi(y)
  \end{tikzcd}
\]%
Here a squiggly arrow \(x \leadsto A\) signifies the equation \(\pi(x) =
A\). The arrow notation is suggestive: the projection \(\pi: X \to C\) defines a
functor \(\widetilde{\pi} : \el(X) \to C\). We prove this now.
\begin{proposition}
  Let \(\widetilde{\pi} : \el(X) \to C\) be the map
  \[%
    x \mapsto \pi(x),\quad \widetilde{f} \mapsto f.
  \]%
  Then \(\widetilde{\pi}\) is a functor.
\end{proposition}
\begin{proof}
  We need to show that \(\widetilde{\pi}\) preserves identities and
  composition. Given an object \(x\) in \(\el(X)\) (i.e., an element
  \(x \in X\)), its identity arrow \(1_x\) is presented by the identity
  \(\widetilde{1_{\pi(x)}}\), so
  \[%
    \pi(1_x) = \pi(\widetilde{1_{\pi(x)}}) = 1_{\pi(x)},
  \]%
  so \(\widetilde{\pi}\) preserves identities. A similar argument works for
  composition.
\end{proof}
The functor \(\widetilde{\pi}\) is what gives a \(C\)\-/indexed set the
structure of a discrete opfibration. A fibration of categories is a
categorification of \(\widetilde{\pi}\), adding more arrows in \(\el(X)\)
besides the ones in \(C\).

We now use \(C\)\-/indexed sets to study the category \(C\). While the objects
of \(C\) may not be faithfully represented by sets, it is true that they are
faithfully represented by \(C\)\-/indexed sets. This property is a
generalization of the embedding of a preorder into its downward closed sets.
\begin{definition}
  Let \(X \in C^{(0)}\) be an object of \(C\). Its \textbf{indexed downset}
  \(\dset X\) is the \(C\)\-/indexed set defined in the following manner.
  \begin{itemize}
  \item The elements of \(\dset X\) are arrows \(f : Y \to X\) in \(C\), for any
    object \(Y\) in \(C\).
  \item The projection \(\pi : \dset X \to C\) sends an arrow \(f : Y \to X\) to
    the object \(Y\).
  \item The multiplication \(\_ \cdot \_ : X \times C^{(1)} \to X\) sends an
    arrow \(f : Y \to X\) and an arrow \(g : Z \to Y\) to the composition \(f
    \circ g\):
    \[%
      f \cdot g \eqdef f \circ g.
    \]%
  \end{itemize}
\end{definition}
\begin{remark}
  The category of elements for the indexed downset \(\dset X\) has a familiar
  name. It is the \textbf{slice category} \(C_{/ X}\). This category is
  sometimes called \(C \downarrow X\), an auspicious hint of the slice
  category's connection to the downsets of a preordered set.
\end{remark}
In order to make sense of how \(C\) is embedded into its \(C\)\-/indexed sets,
we need to make a notion of \(C\)\-/indexed function.
\begin{definition}
  Let \(X\) and \(Y\) be a pair of \(C\)\-/indexed sets. A
  \textbf{\(C\)\-/indexed function} \(\phi : X \to Y\) is a function \(\phi_*\)
  between the underlying sets of \(X\) and \(Y\) which preserves \(\pi\) and
  \(\_\cdot\_\), i.e.,
  \[%
    \pi_Y(\phi_* x) = \pi_X(x), \quad \pwrap{\phi_* x} \cdot_Y f = \phi_* \pwrap{x
      \cdot_X f}.
  \]%
  \(C\)\-/indexed functions may be composed, by composing the underlying
  functions between sets. Furthermore, there is an obvious identity
  function. This gives a category of \(C\)\-/indexed sets, which we call
  \(\iSet{C}\).
\end{definition}
Given an arrow \(f: X \to Y\) in \(C\), we obtain an indexed function \(f_* :
\dset X \to \dset Y\), given by postcomposition:
\[%
  f_*\pwrap{g : Z \to X} \eqdef f \circ g : Z \to Y.
\]%
\begin{lemma}[\(C\)\-/indexed Yoneda lemma]
  The mapping \(X \mapsto \dset X\) and \(f \mapsto f_*\) defines a functor
  \[%
    \dset : C \to \iSet{C}.
  \]%
  Furthermore, this functor is fully faithful. Finally, given an arbitrary
  \(C\)\-/indexed set \(A\), there is a bijection
  \[%
    \iSet{C}\pwrap{\dset X, A} \cong \bwrap{ a \in A \mid \pi(a) = X }.
  \]%
\end{lemma}
\begin{proof}
  We use the only trick in category theory: chase \texttt{refl}.  We first argue
  that \(\dset\) is a functor. Since \(f_*\) is defined by composition in \(C\),
  this is straightforward. We now show that \(\dset\) is faithful.  Let
  \(f, g : X \to Y\) be two arrows in \(C\). Suppose \(f_* = g_*\). We need to
  prove \(f = g\). The identity arrow \(1_X : X \to X\) is an element of
  \(\dset X\). Then
  \[%
    f = f_*(1_X) = g_*(1_X) = g,
  \]%
  so \(\dset\) is faithful. Next we show \(\dset\) is full. Let \(\phi : \dset X
  \to \dset Y\) be an indexed function. Then \(\phi_*(1_X)\) is some arrow \(f :
  X' \to Y\). Since \(\phi\) preserves the projection \(\pi\), \(X' = X\). What
  is left is to argue that \(\phi_* = f_*\). Indeed, for any element \(g: Z \to
  X\) in \(\dset X\),
  \[%
    f_* g = f \circ g = f \cdot g = \phi_*(1_X) \cdot g = \phi_*(1_X \cdot g) =
    \phi_*(g).
  \]%
  This shows \(\phi_* = f_*\). Since \(\phi\) is arbitrary, this shows that
  \(\dset\) is full; hence \(\dset\) is fully faithful.

  Finally, we prove the bijection. Let \(\phi_0 : \dset X \to A\) be an indexed
  function. \({\phi_0}_*(1_X)\) is an element of \(A\); call it
  \(a(\phi_0)\). We will show that \(a : \phi_0 \mapsto a(\phi_0)\) is a
  bijection. We need an inverse to \(a\). Given an element \(a_0 \in A\) such
  that \(\pi(a_0) = X\), we define an indexed function
  \(\phi(a_0) : \dset X \to A\) in the following manner. Given an element
  \(f : Y \to X\) in \(\dset X\), define
  \[%
    \phi(a_0)_*(f) \eqdef a_0 \cdot f \in A.
  \]%
  Since \(\pi(a_0) = X\), this function is well\-/defined, and it is
  straightforward to show that this function preserves \(\pi\) and
  \(\_\cdot\_\). Given an indexed function \(\phi_0 : \dset X \to A\) and an
  element \(f : Y \to X\) in \(\dset X\),
  \[%
    \phi(a(\phi_0))_*(f) = a(\phi_0) \cdot f = {\phi_0}_*(1_X) \cdot f =
    {\phi_0}_*(1_X \cdot f) = {\phi_0}_*(f). 
  \]%
  Thus \(\phi(a(\phi_0)) = \phi_0\). Similarly given an element \(a_0 \in A\)
  such that \(\pi(a_0) = X\),
  \[%
    a(\phi(a_0)) = \phi(a_0)_* (1_X) = a_0 \cdot 1_x = a_0.
  \]%
  Therefore \(a(\phi(a_0)) = a_0\). This shows that \(a(\_)\) and \(\phi(\_)\)
  are inverses, proving the desired bijection.
\end{proof}
We have shown that \(\dset\) is an embedding. When we did this for preorders, we
used this embedding to lift operations on truth\-/values to universal
constructions on a preorder. We do the analogous thing for indexed sets, lifting
operations on sets to universal constructions on a category.

\subsection*{Indexed set theory}

Test

\section{Presheaves: Indexed sets as functors}

\newpage

\bibliographystyle{apalike}
\bibliography{../refs}

\end{document}
