\documentclass[../main.tex]{subfiles}

\begin{document}

\ifSubfilesClassLoaded{%
    \title{Presheaves and Indexed Set Theory}
    \subtitle{}
    \author{}
    \contact{Summer 2024}
    \date{NeuPPL Category Seminar}}{%
    \title{Presheaves and Indexed Set Theory}
    \subtitle{}
    \author{}
    \contact{}
    \date{}}

\makehmtitle%

The first half of these notes developed a categorical perspective on the theory
of preorders. One key step in this theory is the observation that a preorder
\(\pwrap{X,\leq}\) may be treated as a family of indexed propositions
\(\Cl\pwrap{X,\leq}^\op \to \Omega\). This allowed us to lift operations on
truth values \(\Omega\) to universal constructions in a preorder. In a category
\(C\), we can no longer treat objects like indexed \emph{propositions}, but
there is a way to generalize this philosophy. Instead, we need to study indexed
sets, i.e., \emph{presheaves}. Similar to the preorder situation, operations
between sets can be lifted to universal constructions in a category.

\section{A brief account of categories}

A category is a collection of objects and arrows between them which can be
composed. There are myriad situations where such structure arises. Perhaps two
archetypes are interfaces and theories. Qua interface, a category is a
collection of states, called \emph{objects}, and families of transitions between
states, called \emph{arrows}. Functors \(F : C \to \Set\) can be seen as
concrete implementations of this interface. This philosophy is explained well
in~\cite{Fong2019}. This approach also forms the semantics for algebraic
effects. For example, algebras of the \texttt{Writer m} monad, for some monoid
\texttt{m} are described by the category
\begin{figure}[h]
  \centering
  \begin{tikzpicture}[node distance=1.75cm]
    \tikzstyle{place}=[circle,thick,draw=gray!75,fill=gray!20,minimum size=6mm]
    \begin{scope}
      \node [place] (obj) {};
      \draw[thick,->, shorten >=1pt] (obj) to [out=90,in=135,loop,looseness=4.8]
      (obj);
      \draw[thick,->, shorten >=1pt] (obj) to [out=155,in=200,loop,looseness=4.8]
      (obj);
      \draw[thick,->, shorten >=1pt] (obj) to [out=220,in=265,loop,looseness=4.8]
      (obj);
      \node[rotate=45] (dots) at ($ (obj) + (-30:0.5) $) {$\cdots$};
      \draw[thick,->, shorten >=1pt] (obj) to
      [out=375,in=420,loop,looseness=4.8] (obj);
    \end{scope}
  \end{tikzpicture}
\end{figure}
consisting of a single object and an arrow for every element of the monoid
\texttt{m}. Functors from this category to, e.g., \(\Set\) identifies a set
\(X\) (the image of the single object) and a family of functions \(X \to X\) for
every element of \texttt{m}. This is a \texttt{Writer m} algebra.

Qua theory, a category \(C\) is the classifying category of
\emph{some theory}. Functors out of this category are the same as models of this
theory. Given a propositional theory \(T\), we saw that functors from \(\Cl(T)\)
to a preorder \(X\) were the same as models of \(T\) in \(X\). This perspective
is not fundamentally different to the interface perspective.

We will be using categories to describe the logical relations of a programming
language with state, e.g., name allocation or step\-/indexing. We will identify
a category \(C\) whose objects will represent states of a heap. Arrows of \(C\)
will identify how a heap state may be extended with fresh names. We will write
our logical relations as sets indexed by heap states, i.e., functors out of
\(C\) into \(\Set\).

For the sake of completeness, here is a definition of a (small) category.
\begin{definition}
  A \textbf{small category} \(C\) is a pair of sets \(\pwrap{C^{(0)},
    C^{(1)}}\), called its set of \textbf{objects} and its set of
  \textbf{arrows}, respectively, and a family of functions.
  \begin{itemize}
  \item A pair \(s,t : C^{(1)} \to C^{(0)}\) of \textbf{source} and
    \textbf{target} functions. For any arrow \(f \in C^{(1)}\), \(s(f)\) is its
    source object, and \(t(f)\) is its target object. We usually write this as
    \(f : s(f) \to t(f)\).
  \item A function \(1: C^{(0)} \to C^{(1)}\) sending every object \(X \in
    C^{(0)}\) to its \textbf{identity arrow} \(1_X : X \to X\).
  \item A partial function \(C^{(1)} \times C^{(1)} \to C^{(1)}\) called
    \textbf{composition}. \(m(g,f)\) is defined when \(s(g) = t(f)\).
  \end{itemize}
  These functions are required to satisfy two axioms. First, \(m\) must be an
  associative operation:
  \[%
    m(h,m(g,f)) = m(m(h,g),f),
  \]%
  wherever defined. Second, \(1\) must be a left and right identity for \(m\):
  \[%
    m(f,1_{s(f)}) = m(1_{t(f)}, f) = f,
  \]%
  for any \(f \in C^{(1)}\).
\end{definition}
Usually composition \(m(g,f)\) in a category is written as \(g\circ f\) or by
juxtaposition \(gf\).
\begin{definition}
  Given small categories \(C\) and \(D\), a \textbf{functor} \(F: C \to D\) is a
  family of functions \(F^{(0)} : C^{(0)} \to D^{(0)}\) and \(F^{(1)} : C^{(1)}
  \to D^{(1)}\) which preserve the functions \(s,t,1,\) and \(m\).
\end{definition}
Given a category \(C\), swapping the functions \(s\) and \(t\) yields a new
category, \(C^\op\), called its \textbf{opposite category}. This is a
generalization of the relation between \(\leq\) and \(\geq\) for preordered
sets.

There is one more perspective on categories which ties this back to the study on
preorders. Recall, for a preordered set \(\pwrap{X,\leq}\), the classifying
category \(\Cl\pwrap{X,\leq}\) is \emph{thin}---between any two objects \(x,y\)
in \(\Cl\pwrap{X,\leq}\), there is at most one arrow \(x \to y\). A category
\(C\) is an intensional form of a preorder. For a category, it is not enough to
know that there exists an arrow between objects \(X\) and \(Y\), the specific
choice of arrow \(f: X \to Y\) is now relevant. Similarly, \(\Set\) is the
intesional analogue of truth values: it is not enough to know that a set \(X\)
is nonempty; rather, the specific choice of element \(x \in X\) matters. This
perspective tells us that universal constructions ofr categories are
intensional/proof\-/relevant variants of the universal constructions for
preorders. We saw some of these generalizations in the notes on Galois
connections.

\section{Indexed sets}

We begin by describing a presheaf as an indexed set. This is backwards from the
usual order of presentation in a category theory textbook. My goal with starting
with indexed sets is to demonstrate the resemblance between the theory of
presheaves and the theory of sets. This will be done by showing that presheaves
\emph{are} indexed sets. By indexed set, we mean indexed in the following
manner. Given a family of sets \(\bwrap{ X_\alpha \mid \alpha \in I }\), we can
describe this family with a function
\[%
  \pi : X \to I,
\]%
where \(X \eqdef \coprod_{\alpha \in I} X_\alpha\). Given a function \(x \in
X\), \(x\) is contained in some \(X_\alpha\) for a unique \(\alpha\). \(\pi(x)\)
is defined to be this index \(\alpha\). In the case where \(I\) is replaced with
a category \(C\), we follow the same idea, but we introduce some extra structure
to make the set \(X\) compatible with the arrows of \(C\).
\begin{definition}
  Let \(C\) be a small category. A \textbf{\(C\)\-/indexed set} is a set \(X\)
  equipped with a function \(\pi : X \to C^{(0)}\) and a partial function
  \(\mu : X \times C^{(1)} \to X\) satisfying the following constraints.
  \begin{itemize}
  \item \(\mu(x,f)\) is defined whenever \(t(f) = \pi(x)\).
  \item \(\pi\pwrap{\mu(x,f)} = s(f)\).
  \item \(\mu(x,1_{\pi(x)}) = x\).
  \item If \(g \circ f\) is defined, then \(\mu(x,g\circ f) =
    \mu(\mu(x,g),f)\), wherever defined.
  \end{itemize}
  Anticipating an equivalence between \(C\)\-/indexed sets and presheaves on
  \(C\), we could write \(\restr{x}{f}\) in lieu of \(\mu(x,f)\).
\end{definition}
Calling this a \(C\)\-/indexed set is admitedly uncommon. Another name for a
\(C\)\-/indexed set is a \emph{discrete opfibration}. This is more precise, and
it alludes to a more general notion, called a \emph{fibration of categories}. On
the other hand, there is a hint of the notion of \texttt{Writer m} algebra in a
\(C\)\-/indexed set. A category with one object, i.e., a category \(C\) where
\(C^{(0)} \cong \bwrap{\ast}\), is the same structure as a monoid on the set
\(C^{(1)}\). In this sense categories are generalized monoids, where the
multiplication is a partial operation. Given that categories are generalized
monoids, \(C\)\-/indexed sets are generalized modules.
\begin{proposition}
  Let \(C\) be a category with one object. Then a \(C\)\-/indexed set
  \(\pwrap{X,\pi,\mu}\) is the same data as a right\-/action of the monoid
  \(C^{(1)}\) on the set \(X\).
\end{proposition}
\begin{proof}
  Consider a \(C\)\-/indexed set \(\pwrap{X,\pi,\mu}\). Since \(C^{(0)} \cong
  \bwrap{\ast}\), the projection \(\pi : X \to C^{(0)}\) can only be the map
  sending any element of \(X\) to the single element of \(C^{(0)}\). In this
  case, \(\mu : X \times C^{(1)} \to X\) is total. We define the right action on
  \(X\) by \(C^{(1)}\) by the rule
  \[%
    x \cdot f \eqdef \mu(x,f).
  \]%
  Conversely, given a right action of \(C^{(1)}\) on \(X\), we give \(X\) the
  structure of a \(C\)\-/indexed set in the following manner. The projection
  \(\pi\) is unique, since \(C^{(0)}\) is a singleton set. This leaves the map
  \(\mu : X \times C^{(1)} \to X\), which we define using the same rule as
  before.
  \[%
    \mu(x,f) \eqdef x \cdot f.
  \]%
\end{proof}
Due to the above proposition, presheaves are occaisionally called \emph{right
  \(C\)\-/modules}. Section V.7 of \cite{MacLane1994} describes the theory
of right \(C\)\-/modules. Furthermore, to feed mathematicians' preference for
infix notation, we will now write \(x \cdot f\) instead of \(\mu(x,f)\) for any
\(C\)\-/indexed set.

Given a \(C\)\-/indexed set \(\pwrap{X,\pi,\cdot}\), we can lift the categorical
structure on \(C\) to \(X\). In order to give this category a name, we will
anticipate presheaves and abuse notation slightly to call this category
\(\el(X)\), the \emph{category of elements} of \(X\).
\begin{definition}
  Given the \(C\)\-/indexed set \(\pwrap{X,\pi,\cdot}\), its \textbf{category of
    elements} is the following category \(\el(X)\). The objects of \(\el(X)\)
  are the elements of \(X\). Furthermore, there is an arrow
  \(\widetilde{f} : x \to y\) in \(\el(X)\) for every arrow \(f \in C^{(1)}\)
  such that \(x = y \cdot f\). Composition of arrows in \(\el(X)\) is the same
  as composition in \(C\).
\end{definition}
It can be helpful to have a picture for how the arrows in \(\el(X)\) are related
to arrows in \(C\).
\[%
  \begin{tikzcd}
    x \arrow[r,dashed,"\widetilde{f}"] \arrow[d,snake it] & y \arrow[d,snake it] \\
    \pi(x) \arrow[r,"f"] & \pi(y)
  \end{tikzcd}
\]%
Here a squiggly arrow \(x \leadsto A\) signifies the equation \(\pi(x) =
A\). The arrow notation is suggestive: the projection \(\pi: X \to C\) defines a
functor \(\widetilde{\pi} : \el(X) \to C\). We prove this now.
\begin{proposition}
  Let \(\widetilde{\pi} : \el(X) \to C\) be the map
  \[%
    x \mapsto \pi(x),\quad \widetilde{f} \mapsto f.
  \]%
  Then \(\widetilde{\pi}\) is a functor.
\end{proposition}
\begin{proof}
  We need to show that \(\widetilde{\pi}\) preserves identities and
  composition. Given an object \(x\) in \(\el(X)\) (i.e., an element
  \(x \in X\)), its identity arrow \(1_x\) is presented by the identity
  \(\widetilde{1_{\pi(x)}}\), so
  \[%
    \pi(1_x) = \pi(\widetilde{1_{\pi(x)}}) = 1_{\pi(x)},
  \]%
  so \(\widetilde{\pi}\) preserves identities. A similar argument works for
  composition.
\end{proof}
The functor \(\widetilde{\pi}\) is what gives a \(C\)\-/indexed set the
structure of a discrete opfibration. A fibration of categories is a
categorification of \(\widetilde{\pi}\), adding more arrows in \(\el(X)\)
besides the ones in \(C\).

We now use \(C\)\-/indexed sets to study the category \(C\). While the objects
of \(C\) may not be faithfully represented by sets, it is true that they are
faithfully represented by \(C\)\-/indexed sets. This property is a
generalization of the embedding of a preorder into its downward closed sets.
\begin{definition}
  Let \(X \in C^{(0)}\) be an object of \(C\). Its \textbf{indexed downset}
  \(\dset X\) is the \(C\)\-/indexed set defined in the following manner.
  \begin{itemize}
  \item The elements of \(\dset X\) are arrows \(f : Y \to X\) in \(C\), for any
    object \(Y\) in \(C\).
  \item The projection \(\pi : \dset X \to C\) sends an arrow \(f : Y \to X\) to
    the object \(Y\).
  \item The multiplication \(\_ \cdot \_ : X \times C^{(1)} \to X\) sends an
    arrow \(f : Y \to X\) and an arrow \(g : Z \to Y\) to the composition \(f
    \circ g\):
    \[%
      f \cdot g \eqdef f \circ g.
    \]%
  \end{itemize}
\end{definition}
\begin{remark}
  The category of elements for the indexed downset \(\dset X\) has a familiar
  name. It is the \textbf{slice category} \(C_{/ X}\). This category is
  sometimes called \(C \downarrow X\), an auspicious hint of the slice
  category's connection to the downsets of a preordered set.
\end{remark}
In order to make sense of how \(C\) is embedded into its \(C\)\-/indexed sets,
we need to make a notion of \(C\)\-/indexed function.
\begin{definition}
  Let \(X\) and \(Y\) be a pair of \(C\)\-/indexed sets. A
  \textbf{\(C\)\-/indexed function} \(\phi : X \to Y\) is a function \(\phi_*\)
  between the underlying sets of \(X\) and \(Y\) which preserves \(\pi\) and
  \(\_\cdot\_\), i.e.,
  \[%
    \pi_Y(\phi_* x) = \pi_X(x), \quad \pwrap{\phi_* x} \cdot_Y f = \phi_* \pwrap{x
      \cdot_X f}.
  \]%
  \(C\)\-/indexed functions may be composed, by composing the underlying
  functions between sets. Furthermore, there is an obvious identity
  function. This gives a (large) category of \(C\)\-/indexed sets, which we call
  \(\iSet{C}\).
\end{definition}
Given an arrow \(f: X \to Y\) in \(C\), we obtain an indexed function \(f_* :
\dset X \to \dset Y\), given by postcomposition:
\[%
  f_*\pwrap{g : Z \to X} \eqdef f \circ g : Z \to Y.
\]%
\begin{lemma}[\(C\)\-/indexed Yoneda lemma]\label{lemma:yoneda-indexed-sets}
  The mapping \(X \mapsto \dset X\) and \(f \mapsto f_*\) defines a functor
  \[%
    \dset : C \to \iSet{C}.
  \]%
  Furthermore, this functor is fully faithful. Finally, given an arbitrary
  \(C\)\-/indexed set \(A\), there is a bijection
  \[%
    \iSet{C}\pwrap{\dset X, A} \cong \bwrap{ a \in A \mid \pi(a) = X }.
  \]%
\end{lemma}
\begin{proof}
  We use the only trick in elementary category theory: chase \texttt{refl}.  We
  first argue that \(\dset\) is a functor. Since \(f_*\) is defined by
  composition in \(C\), this is straightforward. We now show that \(\dset\) is
  faithful.  Let \(f, g : X \to Y\) be two arrows in \(C\). Suppose
  \(f_* = g_*\). We need to prove \(f = g\). The identity arrow
  \(1_X : X \to X\) is an element of \(\dset X\). Then
  \[%
    f = f_*(1_X) = g_*(1_X) = g,
  \]%
  so \(\dset\) is faithful. Next we show \(\dset\) is full. Let \(\phi : \dset X
  \to \dset Y\) be an indexed function. Then \(\phi_*(1_X)\) is some arrow \(f :
  X' \to Y\). Since \(\phi\) preserves the projection \(\pi\), \(X' = X\). What
  is left is to argue that \(\phi_* = f_*\). Indeed, for any element \(g: Z \to
  X\) in \(\dset X\),
  \[%
    f_* g = f \circ g = f \cdot g = \phi_*(1_X) \cdot g = \phi_*(1_X \cdot g) =
    \phi_*(g).
  \]%
  This shows \(\phi_* = f_*\). Since \(\phi\) is arbitrary, this shows that
  \(\dset\) is full; hence \(\dset\) is fully faithful.

  Finally, we prove the bijection. Let \(\phi_0 : \dset X \to A\) be an indexed
  function. \({\phi_0}_*(1_X)\) is an element of \(A\); call it
  \(a(\phi_0)\). We will show that \(a : \phi_0 \mapsto a(\phi_0)\) is a
  bijection. We need an inverse to \(a\). Given an element \(a_0 \in A\) such
  that \(\pi(a_0) = X\), we define an indexed function
  \(\phi(a_0) : \dset X \to A\) in the following manner. Given an element
  \(f : Y \to X\) in \(\dset X\), define
  \[%
    \phi(a_0)_*(f) \eqdef a_0 \cdot f \in A.
  \]%
  Since \(\pi(a_0) = X\), this function is well\-/defined, and it is
  straightforward to show that this function preserves \(\pi\) and
  \(\_\cdot\_\). Given an indexed function \(\phi_0 : \dset X \to A\) and an
  element \(f : Y \to X\) in \(\dset X\),
  \[%
    \phi(a(\phi_0))_*(f) = a(\phi_0) \cdot f = {\phi_0}_*(1_X) \cdot f =
    {\phi_0}_*(1_X \cdot f) = {\phi_0}_*(f). 
  \]%
  Thus \(\phi(a(\phi_0)) = \phi_0\). Similarly given an element \(a_0 \in A\)
  such that \(\pi(a_0) = X\),
  \[%
    a(\phi(a_0)) = \phi(a_0)_* (1_X) = a_0 \cdot 1_x = a_0.
  \]%
  Therefore \(a(\phi(a_0)) = a_0\). This shows that \(a(\_)\) and \(\phi(\_)\)
  are inverses, proving the desired bijection.
\end{proof}
We have shown that \(\dset\) is an embedding.

\subsection*{Indexed set theory}

When we developed the \(\dset\) embedding for preorders, we used it to lift
operations on truth\-/values to universal constructions on a preorder. We do the
analogous thing for indexed sets, lifting operations on sets to universal
constructions on a category.  Fornotational simplicity, we will use \(\pi\) and
\(\_\cdot\_\) to refer to the projection and multiplication of any
\(C\)\-/indexed set. Given a \(C\)\-/indexed set \(X\), we will write
\(\abs{X}\) for its underlying set, so \(\pi\) is a function
\(\abs{X} \to C^{(0)}\). Finally, we will assume all indexed sets are indexed
over a fixed category \(C\), so we will just call them \emph{indexed sets}
rather than \(C\)\-/indexed sets.

\begin{definition}
  Let \(X\) and \(Y\) be a pair of of indexed sets. The \textbf{product indexed
    set} \(X \times Y\) is defined in the following manner. Its underlying set
  is the restriction
  \[%
    \abs{X \times Y} \eqdef \bwrap{ \pwrap{x,y} \in \abs{X} \times \abs{Y} \mid
      \pi(x) = \pi(y) }.
  \]%
  The projection \(\pi : \abs{X \times Y} \to C^{(0)}\) is derived from the
  projections for \(X\) and \(Y\) (the restriction above ensures this is
  well\-/defined). The multiplication is defined ``pointwise'':
  \[%
    \pwrap{x,y} \cdot f \eqdef \pwrap{x \cdot f, y \cdot f}.
  \]%
\end{definition}
We can write a similar definition for products of arbitraily\-/many indexed
sets.
\begin{definition}
  Let \(X\) and \(Y\) be a pair of indexed sets. The \textbf{disjoint union
    indexed set} \(X \amalg Y\) has underlying set \(\abs{X} \amalg
  \abs{Y}\). The projection \(\pi : \abs{X \amalg Y} \to C^{(0)}\) is the sum
  \(\pi_X \amalg \pi_Y\), same for the multiplication, e.g.,
  \[%
    \texttt{Left} \, x \cdot f \eqdef \texttt{Left} \, \pwrap{x \cdot f}.
  \]%
\end{definition}
As before, we can define arbitrarily\-/large indexed disjoint unions. Refinement
types, i.e., equalizers or pullbacks, look much like products.
\begin{definition}
  Given two indexed functions \(f,g : X \to Y\), the \textbf{equalizer} of \(f\)
  and \(g\) is the indexed set \(E(f,g)\) defined in the following manner. The
  underlying set is the restriction
  \[%
    \abs{E(f,g)} \eqdef \bwrap{ x \in X \mid f(x) = g(x) }.
  \]%
  The projection and multiplication for \(E(f,g)\) is derived from \(X\). This
  is well\-/defined because \(f\) and \(g\) are indexed functions, so they
  commute with projection and multiplication.
\end{definition}
For many constructions, the indexed operations look like the usual
set\-/theoretic ones, this is mostly the case (and it is due to the fact that
limits and colimits of presheaves are computed pointwise; more on that
later). We now discuss indexed subsets.
\begin{definition}
  Given an indexed set \(X\), a subset \(\abs{A} \subseteq \abs{X}\) is a
  \textbf{indexed subset} if it is closed under multiplication. In this case,
  \(\abs{A}\) defines an indexed set \(A\), and the inclusion map \(\abs{A}
  \hookrightarrow \abs{X}\) is an indexed function.
\end{definition}
\begin{definition}
  Given a pair of indexed subsets \(A,B \subseteq X\). Their
  \textbf{intersection} is indexed
  \[%
    \abs{A \cap B} \eqdef \abs{A} \cap \abs{B}.
  \]%
  Similarly, their \textbf{union} is indexed
  \[%
    \abs{A \cup B} \eqdef \abs{A} \cup \abs{B}.
  \]%
\end{definition}
Since the empty set is vacuously an indexed set, the above definitions give the
structure of a (bounded) distributive lattice on the the set \(\Sub{X}\) of
indexed subsets of an indexed set \(X\). This lattice is actually a Heyting
algebra, but the Heyting implication \(\Rightarrow\) is subtle. The above
pattern suggests we might try
\(\abs{A \Rightarrow B} \stackrel{?}{=} \abs{A} \Rightarrow \abs{B}\).  The
issue is the right side is not closed under multiplication. Here is an example.
\begin{tcolorbox}[colframe=NortheasternRed, breakable, enhanced jigsaw]
  \begin{example}
    Let \(C = \Cl\pwrap{\N,\leq}\). Here is a picture of \(C\).
    \[%
      \begin{tikzcd}
        0 \arrow[r,"\partial_0"] & 1 \arrow[r,"\partial_1"] & 2
        \arrow[r,"\partial_2"] & \ldots
      \end{tikzcd}
    \]%
    Here I have named the arrows in \(C\) \(\partial_n : n \to n + 1\)
    (technically, an arbitrary arrow of \(C\) is a composition
    \(\partial_{n-k} \circ \ldots \circ \partial_n : n-k \to n+1\), but this is
    not important for the example). This will make it easier to describe the
    multiplication on \(C\)\-/indexed sets. Speaking intuitively (and loosely),
    a \(C\)\-/indexed set \(X\) is a set which changes over time. The projection
    \(\pi: \abs{X} \to \N\) sends an element \(x \in \abs{X}\) to the stage
    \(n \in \N\) when \(x\) appears. Multiplication by \(\partial_n\) is a map
    \(\_ \cdot \partial_n : \abs{X} \to \abs{X}\) which evolves an element one
    step forward in time. The direction of this evolution is reverse to the
    direction the arrows point in \(C\), so we should understand that looking at
    larger values of \(n \in \N\) amounts to peering further back in time.

    One example of a \(C\)\-/indexed set is a ``descending'' copy of the natural
    numbers \(\widetilde{\N}\):
    \[%
      \abs{\widetilde{\N}} \eqdef C^{(0)} \times \N.
    \]%
    The projection \(\pi: \abs{\widetilde{\N}} \to C^{(0)}\) is defined to be
    projection onto the first component. The multiplication shifts the second
    component downward, if possible:
    \[%
      \pwrap{i+1,n} \cdot \partial_i \eqdef%
      \begin{cases}
        \pwrap{i,n-1} & n \geq 1\\
        \pwrap{i,0} & n = 0
      \end{cases}.
    \]%
    Given a natural number \(a \in \N\), we can form a \(C\)\-/indexed subset of
    \(\widetilde{\N}\):
    \[%
      S_a \eqdef \bwrap{ \pwrap{i,n} \in \abs{\widetilde{\N}} \mid n < a }.
    \]%
    Since the multiplication on \(\widetilde{\N}\) shifts elements downward,
    \(S_a\) is closed under multiplication, so it is in fact an indexed
    subset. The empty set \(\emptyset = S_{0}\) is also an indexed subset. We now
    look at the ordinary \(\Set\)\-/based Heyting implication
    \[%
      \abs{S_a} \Rightarrow \emptyset.
    \]%
    This is the complement of \(S_a\), i.e.,
    \[%
      \abs{S_a} \Rightarrow \emptyset = \bwrap{ x \in \abs{\widetilde{\N}} \mid x
        \geq a}.
    \]%
    This is not closed under multiplication. For example, \(\pwrap{i+1,a}\) is in
    \(\abs{S_a} \Rightarrow \emptyset\), but \(\pwrap{i+1,a} \cdot \partial_i\) is
    not. Thus, \(\abs{S_a} \Rightarrow \emptyset\) is not an indexed subset, so
    the \(\Set\)\-/based Heyting implication is not the correct choice for a
    Heyting implication on indexed subsets!
  \end{example}
\end{tcolorbox}
The above example shows that, in order to define \(\Rightarrow\) among indexed
subsets, we need to force \(\abs{A \Rightarrow B}\) to be closed under
multiplication. The solution? \emph{force} it to be closed under multiplication.
\begin{definition}
  Let \(A,B \subseteq X\) be indexed subsets. The \textbf{Heyting implication}
  of \(A \Rightarrow B\) has underlying subset
  \[%
    \abs{A \Rightarrow B} \eqdef \bwrap{ x \in X \mid \forall f \in C^{(1)}. \,
      x \cdot f \in \abs{A} \Rightarrow x \cdot f \in \abs{B} }.
  \]%
\end{definition}
\begin{proposition}
  As defined above, the Heyting implication \(A \Rightarrow B\) of two indexed
  subsets is closed under multiplication, so it is an indexed subset.
\end{proposition}
\begin{proof}
  Let \(a\) be an element in \(\abs{A \Rightarrow B}\), and let \(f : x \to
  \pi(a)\) be an arrow in \(C\). We need to show that \(a \cdot f\) is in
  \(\abs{A \Rightarrow B}\). To that end, let \(g : y \to x\) be an arrow (note
  \(\pi(a \cdot f) = x\), so any arrow \(g\) for which the multiplication
  \(\pwrap{a \cdot f} \cdot g\) is defined has this shape). We need to show
  that, if \(\pwrap{a \cdot f} \cdot g\) is in \(\abs{A}\), then it is also in
  \(\abs{B}\). Note, \(\pwrap{a \cdot f} \cdot g = a \cdot \pwrap{f \circ
    g}\). Therefore, if \(\pwrap{a \cdot f} \cdot g\) is in \(\abs{A}\), then
  \(a \cdot \pwrap{f \circ g} \in \abs{A}\). Since \(a \in \abs{A \Rightarrow
    B}\), this implies \(a \cdot \pwrap{f \circ g} \in \abs{B}\), i.e.,
  \(\pwrap{a \cdot f} \cdot g \in \abs{B}\). Since \(g\) is arbitrary, this
  shows that \(a \cdot f \in \abs{A \Rightarrow B}\), as desired.
\end{proof}
To show that \(A \Rightarrow B\) is rightfully the Heyting implication, we
should show that it satisfies the same universal property.
\begin{proposition}
  Let \(U, V,\) and \(W\) be a triplet of indexed subsets of \(X\). Then
  \[%
    U \cap V \subseteq W \text{ if and only if } U \subseteq V \Rightarrow W.
  \]%
\end{proposition}
\begin{proof}
  Assume \(U \cap V \subseteq W\). Let \(u \in \abs{U}\). We need to show that
  \(u \in \abs{V \Rightarrow W}\). To that end, let \(f: x \to \pi(u)\) be an
  arrow of \(C\) for which \(u \cdot f\) is defined. Suppose
  \(u \cdot f \in \abs{V}\). Since \(u \in \abs{U}\), and \(\abs{U}\) is closed
  under multiplication, \(u \cdot f \in \abs{U \cap V}\). This is a subset of
  \(W\), so \(u \cdot f \in \abs{W}\). Since \(f\) is arbitrary, this shows \(u
  \in \abs{V \Rightarrow W}\), so \(U \subseteq V \Rightarrow W\). Conversely,
  assume \(U \subseteq V \Rightarrow W\), and let \(u\) be an element of
  \(\abs{U \cap V}\). We need to show that \(u \in \abs{W}\). Consider the arrow
  \(1_{\pi(u)} : \pi(u) \to \pi(u)\). Since \(U \subseteq V \Rightarrow W\), \(u
  \in \abs{V \Rightarrow W}\). We now observe two things:
  \begin{enumerate}
  \item \(u = u \cdot 1_{\pi(u)}\), so \(u \cdot 1_{\pi(u)} \in \abs{V}\).
  \item \(u \cdot 1_{\pi(u)} \in \abs{V}\) implies \(u \cdot 1_{\pi(u)} \in
    \abs{W}\), so \(u \in \abs{W}\).
  \end{enumerate}
  Since \(u\) is arbitrary, this shows \(U \cap V \subseteq W\), as desired.
\end{proof}
\begin{corollary}
  Given an indexed set \(X\), the set of indexed subsets \(\Sub{X}\) (or, less
  ambiguously, \(\Sub_C{X}\)) is a Heyting algebra.
\end{corollary}
The above corollary allows us to interpret propositional logic inside the
category of \(C\)\-/indexed sets and indexed functions. This is done by fixing
an indexed set and interpreting propositions as indexed subsets. Using the
quantifiers as adjoints perspective, we will lift this to an interpretation of
first\-/order logic inside \(C\)\-/indexed sets. Indeed, the category of
\(C\)\-/indexed sets is a \emph{topos}, so we can actually interpret
higher\-/order logic by extending the above theory. To that end, we end this
section by describing exponentials and quotients of indexed sets.
\begin{definition}
  Let \(X\) and \(Y\) be two indexed sets. The \textbf{exponential} indexed set
  \(Y^X\) is defined in the following manner. The elements of \(\abs{Y^X}\) are
  pairs \(\pwrap{v,\phi}\) consisting of an object \(v \in C^{(0)}\) and a
  \(C\)\-/indexed function \(\phi : \dset v \times X \to Y\). The projection \(\pi
  : \abs{Y^X} \to C^{(0)}\) sends a pair \(\pwrap{v,\phi}\) to \(v\).

  The multiplication is defined by the following operation. Given
  \(\pwrap{v,\phi} \in \abs{Y^X}\) and \(f: u \to v\), let \(\phi \ast f\) be the
  indexed function \(\dset y \times X \to Y\)
  \[%
    \pwrap{\phi \ast f}(w \xrightarrow{g} u, x) \eqdef \phi(f \circ g, x). 
  \]%
  We define the multiplication on \(Y^X\) to be
  \[%
    \pwrap{v,\phi} \cdot f \eqdef \pwrap{u,\phi \ast f}.
  \]%
\end{definition}
Like the Heyting implication, we should show that the exponential satisfies the
usual Currying properties we expect from the exponential in the category
\(\Set\).
\begin{proposition}
  Let \(X,Y,\) and \(Z\) be a triplet of indexed sets. There is a bijective
  correspondence between indexed functions
  \[%
    \iSet{C}\pwrap{X \times Y, Z} \cong \iSet{C}\pwrap{X,Z^Y}.
  \]%
\end{proposition}
\begin{proof}
  Let \(\phi: X \times Y \to Z\) be an indexed function. We define the indexed
  function \(\phi^\dagger : X \to Z^Y\) in the following manner. Given an
  element \(x \in \abs{X}\),
  \[%
    \phi^\dagger(x) \eqdef \pwrap{\pi(x), \pwrap{f,y} \mapsto \phi(x \cdot
      f,y)},
  \]%
  where \(f\) is an element of \(\dset \pi(x)\). To show that this is
  well\-/defined, we need to show that \(\pwrap{f,y} \mapsto \phi(x\cdot f, y)\)
  defines an indexed function \(\dset \pi(x) \times Y \to Z\). An element
  \(\pwrap{f,y} \in \dset \pi(x) \times Y\) satisfies \(\pi(f) =
  \pi(y)\). Therefore \(f\) is an arrow of the form \(f : \pi(y) \to \pi(x)\),
  therefore the pair \(\pwrap{x \cdot f, y}\) is in \(X \times Y\), so \(\phi(x
  \cdot f, y)\) is well\-/defined. Since \(\phi\) is an indexed function,
  \(\pwrap{f,y} \mapsto \phi(x\cdot f,y)\) commutes with projection and
  multiplication.

  Having shown \(\phi \mapsto \phi^\dagger\) is a well\-/defined function from
  \(\iSet{C}\pwrap{X \times Y, Z}\) to \(\iSet{C}\pwrap{X,Z^Y}\), we need to
  find an inverse. Let \(\psi : X \to Z^Y\) be an indexed function. Given a pair
  \(\pwrap{x,y} \in \abs{X \times Y}\), \(\psi(x)\) is a pair
  \(\pwrap{v_x,\theta_x}\), where \(\theta_x\) is an indexed function
  \(\dset v \times Y \to Z\). Since \(\psi\) is an indexed function,
  \(\pi(v_x,\theta_x) = \pi(x)\), so \(v_x = \pi(x)\). Furthermore
  \(\pwrap{x,y}\) is assumed to be in \(\abs{X \times Y}\), so
  \(\pi(x) = \pi(y)\). We define \(\psi^\dagger : X \times Y \to Z\) to be the
  function
  \[%
    \psi^\dagger\pwrap{x,y} \eqdef \theta_x\pwrap{1_{\pi(y)},y}.
  \]%
  Since \(v_x = \pi(y)\), this is well\-/defined. Before moving forward, we make
  an observation about \(\theta_x\). Since \(\psi\) is an indexed function, for
  any \(x \in X\) and any \(f: u \to \pi(x)\), \(\psi(x \cdot f) = \psi(x) \cdot
  f\). Expanding this in terms of \(\theta_x\) and \(\theta_{x \cdot f}\) gives
  \[%
    \theta_{x \cdot f} = \theta_x \ast f.
  \]%
  We now need to argue that \(\phi \mapsto \phi^\dagger\) and
  \(\psi \mapsto \psi^\dagger\) are inverses, i.e.,
  \(\phi = \phi^{\dagger\dagger}\) and \(\psi = \psi^{\dagger\dagger}\). Given a
  pair \(\pwrap{x,y} \in \abs{X \times Y}\), we have
  \(\phi^{\dagger\dagger}\pwrap{x,y} = \phi\pwrap{x \cdot 1_{\pi(x)}, y} =
  \phi(x,y)\). Therefore \(\phi^{\dagger\dagger} = \phi\). Finally, given \(x
  \in \abs{X}\),
  \begin{align*}%
    \psi^{\dagger\dagger}(x) &= \pwrap{\pi(x), \pwrap{f,y} \mapsto
      \psi^\dagger(x \cdot f, y)} \\
    &= \pwrap{\pi(x), \pwrap{f,y} \mapsto \theta_{x \cdot f}(1_{\pi(y)}, y)}\\
    &= \pwrap{\pi(x), \pwrap{f,y} \mapsto \theta_x \ast f \pwrap{1_{\pi(y)},
        y}}\\
    &= \pwrap{\pi(x), \pwrap{f,y} \mapsto \theta_x\pwrap{f,y}}\\
    &= \pwrap{\pi(x),\theta_x}\\
    &= \psi(x).
  \end{align*}%
  The last line in this chain of equalities is \(\eta\)\-/reduction. This shows
  \(\psi^{\dagger\dagger} = \psi\), proving the desired bijection.
\end{proof}
Thankfully quotients are simpler.
\begin{definition}
  Let \(X\) be an indexed set, and let \(R\) be an indexed subset of \(X \times
  X\) such that \(\abs{R}\) defines an equivalence relation on \(\abs{X}\). The
  \textbf{quotient} indexed set \(X/R\) has underlying set \(\abs{X} /
  \abs{R}\). The projection and multiplication is inherited from \(X\).
\end{definition}

\begin{remark}
  The discussion on Heyting implication and exponentials is loosely inspired by
  Chapter I of\linebreak\cite{MacLane1994}.
\end{remark}  

\subsection*{From indexed sets to limits}

We have an embedding \(\dset : C \to \iSet{C}\). This allows us to describe
objects of \(C\) using indexed set theory.
\begin{definition}
  Let \(C\) be a small category. Let \(X\) be an indexed set. \(A\) is
  \textbf{representable} if there exists an object \(X \in C^{(0)}\) and an
  indexed bijection \(\dset X \cong A\).
\end{definition}
This notion of representability is what we use to describe limits in \(C\).
\begin{definition}
  Let \(X,Y\) be objects of \(C\). A \textbf{product} of \(X\) and \(Y\) is an
  object \(Z\) and an indexed bijection
  \[%
    \dset Z \cong \dset X \times \dset Y.
  \]%
\end{definition}
We can unpack what this means in terms of arrows. It will help to give the above
bijection a name. Let \(\brak{\_,\_} : \dset X \times \dset Y \to \dset Z\)
denote the inverse of the above bijection. Given an ordered pair \(\pwrap{f,g}
\in \dset X \times \dset Y\), we know \(\pi(f) = \pi(g)\), let \(W\) be this
common object: \(W \eqdef \pi(f)\). Then \(f\) and \(g\) are arrows
\[%
  f : W \to X, \quad g: W \to Y.
\]%
Then, since \(\brak{\_,\_}\) is an indexed function, it preserves the
projection, so \(\brak{f,g}\) is an arrow \(W \to Z\). Since \(1_Z : Z \to Z\)
is an element of \(\dset Z\), there exists a pair \(\pwrap{p_X,p_Y} \in \dset X
\times \dset Y\) such that \(1_Z = \brak{p_X,p_Y}\). This means \(p_X\) and
\(p_Y\) are arrows
\[%
  p_X : Z \to X, \quad p_Y : Z \to Y.
\]%
Since \(\brak{\_,\_}\) preserves multiplication, we obtain a commuting diagram
\[%
  \begin{tikzcd}
    W \arrow[dr,"\exists ! \brak{f,g}", dashed] \arrow[drr, bend left=25,"f"]
    \arrow[ddr, bend right=25, "g", swap] & & \\
    & Z \arrow[d,"p_Y", swap] \arrow[r,"p_X"] & X \\
    & Y. &
  \end{tikzcd}
\]%
This is the usual categorical description of the product. We see from the
definition that a product is unique up to indexed bijection. By the Yoneda lemma
for indexed sets (Lemma~\ref{lemma:yoneda-indexed-sets}), this means that
products are unique up to isomorphism in \(C\). Like for conjunctions in a
preorder, we usually sweep this iunder the rug and refer to the whole
isomorphism class of products by \(X \times Y\).
\begin{definition}
  Given a pair of arrows \(f,g: X \to Y\) in \(C\), an \textbf{equalizer} of
  \(f\) and \(g\) is an object \(Z\) and an indexed bijection
  \[%
    \dset Z \cong E(\dset f, \dset g).
  \]%
\end{definition}
Since \(1_Z\) is an element of \(\dset Z\), the above bijection provides an
element \(e \in E(\dset f, \dset g\) such that \(\pi(e) = Z\). This means
\(e\) is an arrow \(Z \to X\) such that \(f \circ e = g \circ e\).
\begin{warning_box*}
  \begin{exercise}
    Using the indexed bijection, show that the above definition of equalizer is
    equivalent to the following characterization. \(Z\) is an equalizer of \(f\)
    and \(g\) if and only if, for any arrow \(h : V \to X\) such that \(f \circ h
    = g \circ h\), there exists a unique arrow \(\widetilde{h} : E \to Z\) such
    that \(h = e \circ \widetilde{h}\). Another way to state this property is the
    following diagram.
    \[%
      \begin{tikzcd}
        Z \arrow[r,"e"] & X \arrow[r,"f",shift left=1] \arrow[r,"g", shift
        right=1, swap] & Y \\
        V \arrow[ur,"h", swap] \arrow[u, dashed, "\exists ! \widetilde{h}"]
        & & 
      \end{tikzcd}
    \]%
    \nosectionrule
  \end{exercise}
\end{warning_box*}
Having products and equalizers, there is one more type of limit to consider:
\emph{terminal objects}. There is such thing as a terminal indexed set. Let
\(1\) denote the indexed set defined by
\[%
  \abs{1} \eqdef C^{(0)}.
\]%
The projection is the identity, and multiplication is given by
\[%
  X \cdot f \eqdef V,
\]%
where \(f\) is an arrow \(V \to X\).
\begin{proposition}
  For any indexed set \(A\), there is exactly one indexed function \(!_A : A \to
  1\).
\end{proposition}
\begin{proof}
  Indexed functions must preserve projection. Since the projection on \(1\) is
  the identity, this forces
  \[%
    !_A(a) \eqdef \pi(a).
  \]%
\end{proof}
\begin{definition}
  An object \(Z\) of \(C\) is a \textbf{terminal object} if there exists an
  indexed bijection
  \[%
    \dset Z \cong 1.
  \]%
\end{definition}
\begin{warning_box*}
  \begin{exercise}
    From the indexed bijection, show that an object \(Z\) is terminal if and
    only if there exists exactly one arrow \(V \to Z\), for any object \(V \in
    C^{(0)}\).
  \end{exercise}
\end{warning_box*}
We now have the tools needed to define arbitrary limits.

\section{Presheaves: Indexed sets as functors}

\ifSubfilesClassLoaded{%
    \newpage
    \bibliographystyle{apalike}%
    \bibliography{../refs}}%

\end{document}