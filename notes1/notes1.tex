\documentclass{../thesis-note}

\title{Categorical Semantics and Categorical Logic}
\author{NeuPPL Category Seminar}
\date{28 June 2024}

\definecolor{NortheasternRed}{RGB}{200,16,46}

\titlecolor{NortheasternRed}

\newcommand\Prop[1]{\text{{\normalfont{Prop}}}{\pwrap{{#1}}}}
\DeclareMathOperator\Mod{\normalfont Mod}

\usepackage{bussproofs}

\begin{document}

\maketitle%

\section{Introduction: A Proposition for a Metatheory}

Categorical logic and categorical semantics are a generalization of a certain
triangle of structures in propositional logic. First, there is the
\emph{propositional theory} itself. This is the propositional version of an
\emph{equational theory}. It is syntax, and like syntax in a programming
language, we can study transformations between different languages:
\emph{translations}. The second structure is an algebraic model, called the
\emph{classifying category} of the theory. The algebraic model plays the role of
an \emph{equational theory} for the syntax. If we temporarily fast\-/forward to
the simply\-/typed \(\lambda\)\-/calculus, we find an example of structure
appearing in the algebraic model.
\begin{indented}
  \emph{%
    Well\-/typed \(\lambda\)\-/terms do not form a Cartesian\-/closed
    category. However, if we identify terms up to \(\beta\eta\)\-/equivalence,
    the resultant equivalence classes do form a Cartesian\-/closed category.
  }%
\end{indented}
\noindent This is the general idea of the algebraic model: find structure in the
syntax by identifying terms which are proof\-/theoretically the same. The third
structure is the \emph{semantics} of the logic. The semantics is the
``ground\-/truth'' of the logic. We can think of this as observational
equivalence. Semantic truth, i.e., ``\(\vDash\)'' is difficult to reason with,
so we refer to the first two models to explore as much of the semantics as
possible while remaining tractable. For this reason, soundness and completeness
theorems are valuable.
\begin{indented}
  \emph{Soundness:} the first two models are not more powerful than the
  semantics.\\
  \emph{Completeness:} the first two models are at least as powerful as the
  semantics.
\end{indented}
Let us begin by describing this triangle for propositional logic. This will be a
high\-/level overview of the ideas in Chapter 3 of~\cite{Halvorson2019}.

\subsection*{Syntax}

We will work in the classical fragment of propositional logic. Given a set of
atomic propositions \(P,Q,\ldots\), we can form derived propositions \(P \vee Q,
\neg P, P \to Q,\ldots\) using the connectives
\[%
  \wedge, \vee, \Rightarrow, \neg, \top, \bot.
\]%
The meaning of these connectives come from the usual introduction and
elimination rules. These allow us to build Gentzen\-/style proof trees, shown
below.
\begin{gather*}
  \begin{array}{c}
    A \in \Gamma \\\hline \Gamma \vdash A
  \end{array}\quad\quad
  \begin{array}{c}
    \Delta \vdash A \quad \Delta \subseteq \Gamma \\\hline \Gamma \vdash A
  \end{array}\quad\quad
  \begin{array}{c}
    \Gamma \vdash A \quad \Delta, A \vdash B\\\hline \Gamma,\Delta \vdash B
  \end{array}\\
  \begin{array}{c}
    \Gamma \vdash A \quad \Gamma \vdash B \\\hline \Gamma \vdash A \wedge B
  \end{array}\quad\quad
  \begin{array}{c}
    \Gamma \vdash A \wedge B \\\hline \Gamma \vdash A \quad \Gamma \vdash B
  \end{array}\\
  \begin{array}{c}
    \Gamma \vdash A \\\hline \Gamma \vdash A \vee B
  \end{array}\quad\quad
  \begin{array}{c}
    \Gamma \vdash B \\\hline \Gamma \vdash A \vee B
  \end{array}\quad\quad
  \begin{array}{c}
    \Gamma, A \vdash C \quad \Gamma, B \vdash C \\\hline \Gamma, A \vee B \vdash C
  \end{array}\\
  \begin{array}{c}
    \Gamma, A \vdash B \\\hline \Gamma \vdash A \Rightarrow B
  \end{array}\quad\quad
  \begin{array}{c}
    \Gamma \vdash A \Rightarrow B \\\hline \Gamma, A \vdash B
  \end{array}\quad\quad
  \left(
    \begin{array}{c}
      \Gamma \vdash A \Rightarrow B \wedge B \Rightarrow A \\\hline\hline \Gamma \vdash A
      \Leftrightarrow B
    \end{array}
  \right)\\
  \begin{array}{c}
    \Gamma \vdash A \to \bot \\\hline\hline \Gamma \vdash \neg A
  \end{array}\quad\quad
  \begin{array}{c}
    \Gamma \vdash \neg \neg A \\\hline \Gamma \vdash A
  \end{array}\\
  \begin{array}{c}
    \Gamma \vdash A \\\hline \Gamma \vdash \top
  \end{array}\quad\quad
  \begin{array}{c}
    \Gamma \vdash \bot \\\hline \Gamma \vdash A
  \end{array}\\
\end{gather*}
\begin{definition}
  A \textbf{propositional theory} is a pair \(T = \pwrap{\Sigma,\Delta}\)
  consisting of a set \(\Sigma\) of atomic propositions, called the
  \textbf{signature} of the theory, and a set \(\Delta\) of sequents of derived
  propositions of \(\Sigma\), called the \textbf{axioms} of the theory.
\end{definition}
Propositional logic is the study of propositional theories. Using the axioms of
a theory \(T\), we can form a new relation \(\vdash_T\), where \(\Gamma \vdash_T
A\) signifies the existence of a proof\-/tree whose leaves are axioms of \(T\)
such that every internal node is an application of a classical introduction or
elimination rule. One could pause here and study this relation in depth, but we
will keep going. Propositional theories have their own notion of transpiler.
\begin{definition}
  Let \(T_1\) and \(T_2\) be two propositional theories. A \textbf{translation}
  \(F: T_1 \to T_2\) is a map sending atomic propositions in \(\Sigma_1\) to
  derived propositions of \(\Sigma_2\). By declaring that this map preserves the
  logical connectives, this extends to a map from derived propositions of
  \(\Sigma_1\) to derived propositions of \(\Sigma_2\). We require that \(F\)
  satisfy the additional constraint:
  \[%
    \Gamma \vdash_{T_1} A \implies F\Gamma \vdash_{T_2} F A.
  \]%
\end{definition}
We are often interested in asking whether two theories \(T_1\) and \(T_2\) have
the same level of expressivity and proving power. Given that we have a notion of
arrow between theories, we might try to study \emph{isomorphisms}. However,
the constraint that \(GF = \id : T_1 \to T_1\) for a pair of translations \(F:
T_1 \to T_2\) and \(G: T_2 \to T_1\) is often too restrictive. For one, atomic
propositions must be sent to themselves, but the image of a derived proposition
under a translation is never atomic. Therefore, \(F\) must send any atomic
proposition to an atomic proposition. If we have an isomorphism, the same must
be true for \(G\). This shows that any isomorphism of propositional theories
must be just a relabeling of the symbols. Not only that, theories also have
axioms, and isomorphisms do not care about the axioms. We are led to a weakening
of isomorphism which is still good enough to model the logic inside a
propositional theory.
\begin{definition}
  Two translations \(F, F' : T_1 \to T_2\) are \textbf{provably equivalent} or
  \textbf{homotopic} in the case that, for any \(T_1\)\-/proposition \(\phi\),
  \[%
    \cdot \vdash_{T_2} F\phi \Leftrightarrow F' \phi.
  \]%
  An \textbf{equivalence} of theories is a pair of translations \(F: T_1 \to
  T_2\) and \(G: T_2 \to T_1\) such that \(GF\) is provably equivalent to
  \(\id_{T_1}\) and \(FG\) is provably equivalent to \(\id_{T_2}\), i.e.,
  for any \(T_1\) proposition \(\phi\) and \(T_2\) proposition \(\psi\),
  \[%
    \cdot \vdash_{T_1} \phi \Leftrightarrow GF \phi \text{ and } \cdot
    \vdash_{T_2} \psi \Leftrightarrow FG \psi.
  \]%
\end{definition}
If a translation \(F: T_1 \to T_2\) is part of an equivalence, then \(F\)
resembles a \emph{fully abstract} compiler. Our metatheory of translations
should explain when equivalences exist, and it should provide methods to
construct them.

\subsection*{Semantics}

I mentioned that equivalences are good enough to model the logic inside a
propositional theory. Let's take a moment to formalize what is meant by
``modeling the logic'' inside a theory. For classical propositional logic, the
only thing that matters is truth tables, essentially. Let \(\Omega \eqdef
\bwrap{\True, \False}\) be the set of classical truth values.
\begin{definition}
  Let \(\Sigma\) be a propositional signature. An \textbf{interpretation} (or,
  if you want, a \textbf{denotational semantics}) for the signature \(\Sigma\)
  is a map \(\fbb{\cdot}\) from atomic propositions in \(\Sigma\) to truth
  values \(\Omega\)
\end{definition}
We can extend the map from a denotational semantics \(\fbb\cdot : \Sigma \to
\Omega\) to derived propositions in \(\Sigma\). We do this by sending the
logical connectives to the appropriate boolean operations
\begin{gather*}
  \fbb{A \wedge B} \eqdef \fbb{A} \,\texttt{and}\, \fbb{B} \quad\quad
  \fbb{A \vee B} \eqdef \fbb{A} \,\texttt{or}\, \fbb{B} \quad\quad
  \fbb{A \Rightarrow B} \eqdef
  \texttt{if}\,\fbb{A}\,\texttt{then}\,\fbb{B}\,\texttt{else}\,\True\\
  \fbb{\neg A} \eqdef \texttt{not}\,\fbb{A}\quad\quad
  \fbb{\top} \eqdef \True \quad\quad
  \fbb{\bot} \eqdef \False.
\end{gather*}
\begin{definition}
  Let \(T\) be a propositional theory with signature \(\Sigma\) and axioms
  \(\Delta\). A \textbf{model} of the theory \(T\) is an interpretation
  \(\fbb\cdot\) of \(\Sigma\) such that, for any sequent \(\phi \vdash \psi\) in
  \(\Delta\), \( \fbb{\phi} \leq \fbb{\psi}\), where the ordering \(\leq\) is
  the usual ordering on \(\Omega\), interpreted as a boolean algebra.  If we
  give the model a name, say, \(M : \Sigma \to \Omega\), then we define the
  notation \(\phi \vDash_M \psi\) to mean \(\fbb{\phi}^M \leq \fbb{\psi}^M\).
\end{definition}
We can treat the semantics as the ground\-/truth of the propositional theory. We
are concerned with which propositions are true, and the classical introduction
and elimination rules help us explore the space of true propositions by
identifying operations which preserve the ground truth. This is a soundness
theorem.
\begin{theorem}[Soundness theorem]
  Let \(\fbb{\cdot}^M\) be a model of a propositional theory \(T\). Suppose
  \(\phi \vdash_T \psi\) for some propositions \(\phi\) and \(\psi\). Then
  \(\phi \vDash_M \psi\).
\end{theorem}

Similarly, our notion of translation can also be justified by the
semantics. Consider a translation \(F: T_1 \to T_2\), and suppose
\(\fbb{\cdot}^M\) is a model of \(T_2\). We obtain a model of
\(\fbb{\cdot}^{F^* M}\) of \(T_1\), given by composition, essentially
\[%
  \Sigma_1 \xrightarrow{F} \text{Prop}(\Sigma_2) \xrightarrow{\fbb{\cdot}^M}
  \Omega.
\]%
\begin{proposition}
  \(\fbb{\cdot}^{F^* M}\) is indeed a model of \(T_1\).
\end{proposition}
\begin{proof}
  We need to show that \(F^*\fbb{\cdot}\) preserves the axioms of \(T_1\). Let
  \(\phi \vdash \psi\) be such an axiom. Since \(F\) is a translation, \(F\phi
  \vdash_{T_2} F\psi\). By the soundness theorem, \(F\phi \vDash_M
  F\psi\). Finaly, observe that
  \[%
    \fbb{\phi}^{F^* M} \eqdef \fbb{F\phi}^M.
  \]%
  Therefore \(F\phi \vDash_M F\psi\) implies \(\phi \vDash_{F^* M} \psi\), as
  desired.
\end{proof}
Recall that our notion of translation was too strict to have interesting
inverses. Here is another reason why we need to consider ``homotopic''
translations.
\begin{proposition}
  Let \(F, G : T_1 \to T_2\) be a pair of translations. If \(F\) and \(G\) are
  homotopic, then, for any model \(\fbb{\cdot}^M\) of \(T_2\), we have, for any
  derived proposition \(\phi\) of \(T_1\),
  \[%
    \fbb{\phi}^{F^* M} = \fbb{\phi}^{G^* M} : \Omega.
  \]%
\end{proposition}
\begin{proof}
  Since \(F\) and \(G\) are homotopic, the sequent \(\cdot \vdash_{T_2} F\phi
  \leftrightarrow G\phi\) is provable in \(T_2\). Since \(M\) is a model, this
  means
  \begin{lstlisting}[language=ml,escapeinside={(*}{*)}]
    if True
    then ((if (*$\fbb{F\phi}^M$*) then (*$\fbb{G\phi}^M$*) else True)
          and
          (if (*$\fbb{G\phi}^M$*) then (*$\fbb{F\phi}^M$*) else True))
    else
      True      
  \end{lstlisting}
  evaluates to \(\True\). Furthermore, the laws of the boolean operations imply
  that the above expression is \(\True\) if and only if
  \(\fbb{F\phi}^M \leq \fbb{G\phi}^M\) and vice versa. Antisymmetry now implies
  \(\fbb{\phi}^{F^* M} = \fbb{\phi}^{G^* M}\), as desired.
\end{proof}
In fact, we could have defined two translations to be homotopic in the case that
they are indistinguishable in any model. This was the original approach, when
homotopy of translations was defined in~\cite{Ahlbrandt1986}. Proving that this
is equivalent to the syntax\-/focused approach mentioned above requires a
completeness theorem for propositional theories.

\begin{remark}
  To make sense of an interpretation and model, we only need a notion of truth
  value and boolean operations on truth values. Therefore, we could replace
  \(\Omega\) with any boolean algebra \(B\) and define a notion of
  \textbf{\(B\)\-/valued model} \(\Sigma \to B\). All the results in this
  passage transfer to \(B\)\-/valued models.
\end{remark}

\subsection*{Algebra}

We now present the third side of the triangle. In the case of propositional
logic, the equational theory is trivial. However the change in perspective from
introducing category theory introduces the key ideas in categorical logic, and
it shows that a propositional theory is fully subsumed by its classifying
category.

\begin{definition}
  Let \(T\) be a propositional theory. Notice that the set of derived
  propositions \(\Prop{T}\) has a preorder, given by \(\vdash_T\):
  \[%
    \phi \leq \psi \text{ if and only if } \phi \vdash_T \psi.
  \]%
  In the warm\-/up we saw that we can define the classifying category of a
  preorder. The \textbf{classifying category} of the theory \(T\) is the
  category
  \[%
    \Cl\pwrap{T} \eqdef \Cl\pwrap{\Prop{T},\vdash_T}.
  \]%
  Explicitly, the objects of \(\Cl\pwrap{T}\) are derived propositions in
  \(\Prop{T}\), and there exists a unique arrow \(\phi \to \psi\) in
  \(\Cl\pwrap{T}\) in the case that \(\phi \vdash_T \psi\).
\end{definition}
I am choosing to call this category the classifying category of \(T\),
following~\cite{Jacobs1999}. There are many other names for this category;
another common name is the \emph{syntactic category} of the theory \(T\). The
rest of this section will describe how a propositional theory can be described
in terms of classifying categories.

For example, a common thing to ask when given a category is \emph{what are its
  isomorphisms?} Since there is at most one arrow between two objects in
\(\Cl\pwrap{T}\), the only arrow \(\phi \to \phi\) must be the identity arrow,
for any object \(\phi\) in \(\Cl\pwrap{T}\). This means two objects \(\phi\) and
\(\psi\) are isomorphic in \(\Cl\pwrap{T}\) if and only if there exist arrows
\(\phi \to \psi\) and \(\psi \to \phi\). Unpacking the meaning of an arrow in
\(\Cl\pwrap{T}\), we obtain the following proposition.
\begin{proposition}
  Let \(T\) be a propositional theory, and let \(\phi\) and \(\psi\) be two
  objects of \(\Cl\pwrap{T}\), i.e., derived propositions of \(T\). \(\phi\) and
  \(\psi\) are isomorphic in \(\Cl\pwrap{T}\) if and only if \(\phi \vdash_T
  \psi\) and \(\psi \vdash_T \phi\), equivalently,
  \[%
    \cdot \vdash_T \phi \Leftrightarrow \psi.
  \]%
\end{proposition}

\subsubsection*{Translations}

We can describe translations as functors.
\begin{proposition}
  Let \(F: T_1 \to T_2\) be a translation between propositional theories. Then
  \(F\) defines a mapping \(\Prop{T_1} \to \Prop{T_2}\) which is
  monotonic. Therefore \(F\) defines a functor
  \[%
    \Cl\pwrap{F} : \Cl\pwrap{T_1} \to \Cl\pwrap{T_2}.
  \]%
  Furthermore, two translations \(F_1, F_2: T_1 \to T_2\) are provably
  equivalent (i.e., homotopic) if and only if \(\Cl\pwrap{F_1}\) and
  \(\Cl\pwrap{F_2}\) are naturally isomorphic.
\end{proposition}
\begin{proof}
  The statement that \(F\) is a monotonic map means, for any pair of derived
  propositions \(\phi,\psi \in \Prop{T_1}\),
  \[%
    \phi \vdash_{T_1} \psi \text{ implies } F\phi \vdash_{T_2} F\psi.
  \]%
  This is part of the definition of a translation. In the warm\-/up, we showed
  that a monotonic map between preordered sets defines a functor between their
  classifying categories. An object of \(\Cl\pwrap{T_1}\) is a derived
  proposition \(\phi \in \Prop{T_1}\); we define
  \[%
    \Cl\pwrap{F}\phi \eqdef F\phi.
  \]%
  We move to the final part of the proposition. Assume \(F_1\) and \(F_2\) are
  provably equivalent translations from \(T_1\) to \(T_2\). The statement that
  \(\Cl\pwrap{F_1}\) and \(\Cl\pwrap{F_2}\) are naturally isomorphic unpacks to
  the following. For any arrow \(\phi \to \psi\) in \(\Cl\pwrap{T_1}\), there
  exists arrows \(F_1 \phi \to F_2 \phi\), \(F_1 \psi \to F_2 \psi\), \(F_2 \phi
  \to F_1 \phi\), and \(F_2 \psi \to F_1 \psi\) fitting into a pair of commuting
  squares in \(\Cl\pwrap{T_2}\).
  \[
    \begin{array}{cc}
      \begin{tikzcd}
        F_1 \phi \arrow[r] \arrow[d] & F_1 \psi \arrow[d] \\
        F_2 \phi \arrow[r] & F_2 \psi
      \end{tikzcd} &
      \begin{tikzcd}
        F_2 \phi \arrow[r] \arrow[d] & F_2 \psi \arrow[d] \\
        F_1 \phi \arrow[r] & F_1 \psi.
      \end{tikzcd}
    \end{array}
  \]%
  We now need to prove that these squares exist. Let \(\phi \to \psi\) be an
  arrow in \(\Cl\pwrap{T_1}\), so \(\phi \vdash_T \psi\). Since \(F_1\) and
  \(F_2\) are translations, the following sequents are provable in \(T_2\):
  \begin{gather*}
    F_1 \phi \vdash_{T_2} F_1 \psi \quad F_2 \phi \vdash_{T_2} F_2 \psi.
  \end{gather*} 
  Since \(F_1\) and \(F_2\) are provably equivalent, the following sequents are
  provable in \(T_2\):
  \begin{gather*}%
    F_1 \phi \vdash_{T_2} F_2 \phi \quad F_2 \phi \vdash_{T_2} F_1 \phi\\
    F_1 \psi \vdash_{T_2} F_2 \psi \quad F_2 \psi \vdash_{T_2} F_1 \psi.
  \end{gather*}%
  These six sequents define six arrows in \(\Cl\pwrap{T_2}\). These six arrows
  can be arranged to make the sides in two commuting squares above. Thus,
  \(\Cl\pwrap{F_1}\) and \(\Cl\pwrap{F_2}\) are naturally isomorphic.

  Conversely, assume that \(\Cl\pwrap{F_1}\) and \(\Cl\pwrap{F_2}\) are
  naturally isomorphic. Then for any arrow \(\phi \to \psi\) in
  \(\Cl\pwrap{T_1}\) we can find two commuting squares of the form shown
  above. Let \(\phi\) be an arbitrary derived proposition in \(\Prop{T_1}\),
  i.e., an arbitrary object of \(\Cl\pwrap{T_1}\). If we apply this to the
  identity arrow \(\phi \to \phi\), then we obtain arrows
  \(F_1 \phi \to F_2 \phi\) and \(F_2 \phi \to F_1 \phi\) (these are the
  vertical sides of the two squares). Unpacking the meaning of an arrow in
  \(\Cl\pwrap{T_2}\), this means
  \[%
    F_1 \phi \vdash_{T_2} F_2\phi \quad F_2\phi \vdash_{T_2} F_1 \phi.
  \]%
  Since \(\phi\) is arbitrary, this means \(F_1\) and \(F_2\) are provably
  equivalent.
\end{proof}
Recall that two translations \(F: T_1 \to T_2\) and \(G: T_2 \to T_1\) define an
equivalence of theories in the case that \(GF\) and \(FG\) are provably
equivalent to \(\id_{T_1}\) and \(\id_{T_2}\). The above proposition determines
when two translations are equivalent in terms of their associated functors. This
gives us the following corollary.
\begin{corollary}
  Let \(F: T_1 \to T_2\) and \(G: T_2 \to T_1\) be a pair of translations. \(F\)
  and \(G\) define an equivalence of theories if and only if \(\Cl\pwrap{F}\) and
  \(\Cl\pwrap{G}\) define an equivalence of categories
  \begin{gather*}
    \Cl\pwrap{F} : \Cl\pwrap{T_1} \to \Cl\pwrap{T_2}\\
    \Cl\pwrap{G} : \Cl\pwrap{T_2} \to \Cl\pwrap{T_1}.
  \end{gather*}
\end{corollary}
We can actually strengthen the above result on translations and fully identify
which functors \(\Cl\pwrap{T_1} \to \Cl\pwrap{T_2}\) come from translations
\(T_1 \to T_2\). This requires a finer description of the structure of
\(\Cl\pwrap{T}\), which we will get to later. Let's just state the result now.
\begin{proposition}
  Let \(F: T_1 \to T_2\) be a translation. Then
  \(\Cl\pwrap{F}: \Cl\pwrap{T_1} \to \Cl\pwrap{T_2}\) satisfies the following
  properties.
  \begin{itemize}
  \item \(\Cl\pwrap{F}\) preserves finite limits (i.e., infima);
  \item \(\Cl\pwrap{F}\) preserves finite colimits (i.e., suprema);
  \item \(\Cl\pwrap{F}\) preserves exponentials (i.e., \(\Rightarrow\)).
  \end{itemize}
  Furthermore, any functor \(\Cl\pwrap{T_1} \to \Cl\pwrap{T_2}\) satisfying
  these properties is described by a translation \(T_1 \to T_2\).
\end{proposition}

\subsubsection*{Interpretations}

Perhaps what is more surprising is that interpretations of a propositional
theory \(T\) into a boolean algebra \(B\) can also be described as a functor. A
boolean algebra has a canonical partial order associated to it: for any elements
\(a,b\) in a boolean algebra \(B\), we write \(a \leq b\) in the case that \(a
\wedge b = a\). A partial order is a special kind of preorder, so we can refer
to the classifying category of a boolean algebra.
\begin{definition}
  Given a boolean algebra \(B\), the \textbf{classifying category} of \(B\) is
  the classifying category of the canonical partial order on \(B\):
  \[%
    \Cl\pwrap{B} \eqdef \Cl\pwrap{B,\leq}.
  \]%
\end{definition}
Similar to the story with translations, any boolean algebra homomorphism \(F:
B_1 \to B_2\) defines a functor \(\Cl\pwrap{F} : \Cl\pwrap{B_1} \to
\Cl\pwrap{B_2}\), since the set of objects of \(\Cl\pwrap{B}\) are the elements
of \(B\).

On the other hand, models are also functors. Given a propositional
theory \(T\) and a \(B\)\-/valued model \(\fbb{\cdot}^M : T \to B\), we
obtain a functor \(M : \Cl\pwrap{T} \to \Cl\pwrap{B}\) by setting
\[%
  M\pwrap{\phi : \Cl\pwrap{T}} \eqdef \fbb{\phi : \Prop{T}}^M.
\]%
\begin{proposition}
  \(M\) is indeed a functor from \(\Cl\pwrap{T}\) to \(\Cl\pwrap{B}\).
\end{proposition}
\begin{proof}
  We need to verify that \(M\) sends arrows of \(\Cl\pwrap{T}\) to arrows of
  \(\Cl\pwrap{B}\). Let \(\phi \to \psi\) be an arrow in \(\Cl\pwrap{T}\). Since
  \(\fbb{\cdot}^M\) is a model, this implies
  \[%
    \fbb{\phi}^M \leq \fbb{\psi}^M.
  \]%
  Therefore, there exists an arrow \(\fbb{\phi}^M \to \fbb{\psi}^M\) in
  \(\Cl\pwrap{B}\). Since \(M(\phi) = \fbb{\phi}^M\), this is an arrow \(M(\phi)
  \to M(\psi)\), as desired.
\end{proof}
Given a translation \(F: T_1 \to T_2\), recall we obtained a ``pull\-/back''
operation
\[%
  F^* : \Mod(T_2,B) \to \Mod(T_1,B),
\]%
where \(\Mod(T,B)\) denotes the set of \(B\)\-/valued models of the theory
\(T\). From the perspective of the classifying categories, this operation is
just composition.
\begin{theorem}
  Let \(F: T_1 \to T_2\) be a translation between propositional theories. Let
  \(\fbb{\cdot}^M : T_2 \to B\) be a \(B\)\-/valued model of \(T_2\). Then
  \(\fbb{\cdot}^{F^* M} : T_1 \to B\) is the pulled\-/back model of \(T_1\)
  under \(F\). From these we obtain functors
  \begin{gather*}
    \Cl\pwrap{F} : \Cl\pwrap{T_1} \to \Cl\pwrap{T_2}\\
    M : \Cl\pwrap{T_2} \to \Cl\pwrap{B} \\
    F^* M: \Cl\pwrap{T_1} \to \Cl\pwrap{B}.
  \end{gather*}
  These functors fit into a commuting diagram
  \[%
    \begin{tikzcd}
      \Cl\pwrap{T_2} \arrow[r,"M"] & \Cl\pwrap{B} \\
      \Cl\pwrap{T_1}. \arrow[u,"\Cl\pwrap{F}"] \arrow[ur,dashed, "F^*M", swap] &
    \end{tikzcd}
  \]%
  In other words, \(F^* M = M \circ \Cl\pwrap{F}\).
\end{theorem}
The key observation behind all of the proofs so far is that the classifying
category of a theory is almost a copy of the theory. We will see later a
construction, called the \emph{internal logic} of a category, which will recover
the original theory from its classifying category.

%\subsubsection*{Finer structure}

\newpage
\bibliographystyle{apalike}
\bibliography{../refs}
\newpage
\appendix
\section{The Lindenbaum Algebra}

The Lindenbaum algebra is another algebraic model of a propositional theory. We
saw that, for the purposes of understanding the semantics of a theory, we only
need equivalence classes of translations up to homotopy/provable
equivalence. This was another point in a pattern of observations that the syntax
of propositional logic hides certain symmetries which appear from the
introduction and elimination rules. The Lindenbaum algebra of a propositional
theory is a quotient
\[%
  \faktor{\text{Propositions}}{\text{Proof theory}}.
\]%
This is the simplest form of an \emph{equational theory}. Traditionally, the
algebraic semantics was given by a boolean algebra of propositions modulo the
double\-/sided implication \(\leftrightarrow\).
\begin{definition}
  Let \(T\) be a classical propositional theory. The \textbf{Lindenbaum algebra}
  is the boolean algebra \(\pwrap{L(T),\leq}\) of equivalence classes of
  propositions \(\fb{\phi}\), where two propositions \(\phi\) and \(\phi'\) are
  in the same class in the case
  \[%
    \phi \vdash_T \phi' \text{ and } \phi' \vdash_T \phi.
  \]%
  The ordering \(\leq\) is given by
  \[%
    \fb\phi \leq \fb\psi \stackrel{\text{def}}{\equiv} \phi \vdash_T \psi.
  \]%
\end{definition}
There are a few nice properties of the Lindenbaum algebra, which highlights the
framework in \cite{Halvorson2019}.
\begin{proposition}
  Let \(T\) be a propositional theory. The mapping \(\phi \mapsto \fb{\phi}\)
  defines a \(L(T)\)\-/valued model of \(T\). This mapping is called the
  \textbf{canonical interpretation} of \(T\) in \(L(T)\).
\end{proposition}
\begin{proposition}
  There is a bijective correspondence between provable\-/equivalence classes of
  translations \(F: T_1 \to T_2\) and boolean algebra homomorphisms
  \(L(T_1) \to L(T_2)\), fitting into a square, where the vertical arrows are
  the canonical interpretations.
  \[%
    \begin{tikzcd}
      T_1 \arrow[r,"F"] \arrow[d] & T_2 \arrow[d] \\
      L(T_1) \arrow[r,dashed] & L(T_2).
    \end{tikzcd}
  \]%
\end{proposition}
\begin{proposition}
  Let \(T\) be a propositional theory and \(B\) a boolean algebra. There is a
  bijective correspondence between \(B\)\-/valued models of \(T\) and boolean
  algebra homomorphisms \(L(T) \to B\), related by the canonical interpretation
  \[%
    \begin{tikzcd}
      L(T) \arrow[r,dashed] & B\\
      T. \arrow[u] \arrow[ur] &
    \end{tikzcd}
  \]%
\end{proposition}
For classical propositional logic, the Lindenbaum algebra is
sufficient. However, our goal is to apply categorical logic to programming
languages, where the proof theory is not so simple. I mentioned that the
Lindenbaum algebra \(L(T)\) is the quotient of \(T\) by
\(\leftrightarrow\). Under the Curry\-/Howard interpretation, this shows that
the Lindenbaum algebra \(L(T)\) identifies isomorphic types!

For this reason, it will be helpful to take a step backwards and permit
ourselves to provide our own equational theory. This will require more
complicatd algebra, but it will build a system which is amenable to the
Curry\-/Howard correspondence.

\end{document}
