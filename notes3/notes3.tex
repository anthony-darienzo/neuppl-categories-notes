\documentclass{../thesis-note}

\title{Galois Connections}
\author{NeuPPL Category Seminar}
\date{Summer 2024}

\definecolor{NortheasternRed}{RGB}{200,16,46}

\titlecolor{NortheasternRed}

\newcommand\Prop[1]{\text{{\normalfont{Prop}}}{\pwrap{{#1}}}}
\newcommand\Dwd{\text{\normalfont Dwd}}
\newcommand\natleq{\mathrel{\stackrel{\text{nat}}{\leq}}}
\DeclareMathOperator\dset{\downarrow}
\DeclareMathOperator\Mod{\normalfont Mod}
\newcommand\Pcal{{\mathcal{P}}}

\usepackage{bussproofs}

\begin{document}

\maketitle%

This time we return to~\cite{Fong2019} to discuss a special kind of relation
between preorders: \emph{Galois connections}. These are the preorder\-/specific
forms of an adjunction between categories.

\section{Definition of a Galois connection}

The setting of a Galois connection is a pair of preorders \(\pwrap{X,\leq}\) and
\(\pwrap{Y,\leq}\). In what follows, I will write \(X\) for the pair
\(\pwrap{X,\leq}\), etc., using the same symbol for the order\-/relation in any
preorder.
\begin{definition}
  A \textbf{Galois connection} between the preordered sets \(X\) and \(Y\) is a
  pair of monotone maps \(f: X \to Y\) and \(g: Y \to X\) satisfying the
  following property. For any pair of elements \(x \in X\) and \(y \in Y\),
  \[%
    \begin{array}{rl}
      f(x) &\leq y \\\hline\hline
      x &\leq g(y).
    \end{array}
  \]%
  In this case \(f\) is called the \textbf{left adjoint}, and \(g\) is called
  the \textbf{right adjoint}. We express the relationship between \(f\) and
  \(g\) by writing \(f \dashv g\).
\end{definition}
One powerful property of Galois connections (and adjunctions, more generally) is
there preservation of limits and colimits.
\begin{proposition}\label{prop:adjoints-preserve-limits}
  Let \(f: X \to Y\) and \(g: Y \to X\) form a Galois connection \(f \dashv g\)
  between preordered sets \(X\) and \(Y\). Then \(f\) preserves all colimits (i.e.,
  suprema), and \(g\) preserves all limits (i.e., infima).
\end{proposition}
\begin{remark}
  A functor which preserves colimits is sometimes called a \emph{right\-/exact
    functor}, and a functor which preserves limits is sometimes called a
  \emph{left\-/exact functor}. Therefore the above proposition can be restated
  more mnemonically as: \emph{if a functor has a right adjoint, then it
    is right exact, and if it has a left adjoint, then it is left exact.} 
\end{remark}
\begin{proof}[Proof of Proposition~\ref{prop:adjoints-preserve-limits}]
  We will show that \(f\) preserves suprema. The proof that \(g\) preserves
  infima is formally dual. Consider a diagram \(J : D \to
  X\). Assuming the colimit along \(J\) exists, we need to show
  \[%
    f\pwrap{ \bigvee_{i \in D} J(i) } \cong \bigvee_{i \in D} f(J(i)).
  \]%
  To that end, let \(x\) be an element in \(X\) representing \(\bigvee_{i \in D}
  J(i)\). We will show that \(f(x)\) is the least upper bound of the set
  \(\bwrap{ f(J(i)) \mid i \in D }\). Since \(J(i) \leq x\) for any \(i \in D\),
  \(f(J(i)) \leq f(x)\) since \(f\) is order\-/preserving. Therefore \(f(x)\) is
  an upper bound. On the other hand, suppose \(y \in Y\) is an element
  satisfying \(f(J(i)) \leq y\) for any \(i \in D\). Since \(f\) is left adjoint
  to \(g\), this implies \(J(i) \leq g(y)\) for any \(i \in D\). Since \(x\) is
  a least upper bound of \(\bwrap{J(i) \mid i \in D}\), \(x \leq g(y)\). We use
  the Galois connection again to deduce from this that \(f(x) \leq
  y\). Therefore \(f(x)\) is a least upper bound, as desired.
\end{proof}
We now apply the fundamental trick in category theory: use
\(\texttt{refl}\). Every preorder is reflexive, so \(f(x) \leq f(x)\), for any
\(x \in X\). By the above property of Galois connections
\[%
  \begin{array}{rl}
    f(x) &\leq f(x) \\\hline
    x &\leq g(f(x)).
  \end{array}
\]%
Similarly, \(g(y) \leq g(y)\), so the Galois connection implies
\[%
  \begin{array}{rl}
    g(y) &\leq g(y) \\\hline
    f(g(y)) &\leq y.
  \end{array}
\]%
In fact, these properties characterize a Galois connection.
\begin{theorem}\label{thm:galois-connection-closure}
  Let \(f: X \to Y\) and \(g: Y \to X\) be a pair of order\-/preserving
  maps. \(f\) and \(g\) form a Galois connection \(f \dashv g\) if and only if,
  for any \(x \in X\) and \(y \in Y\),
  \[%
    x \leq g(f(x)) \text{ and } f(g(y)) \leq y.
  \]%
\end{theorem}
\begin{warning_box*}
  The conditions \(x \leq g(f(x))\) and \(f(g(y)) \leq y\) for any \(x \in X\)
  and \(y \in Y\) show that \(g \circ f\) is a \emph{closure operator} and \(f
  \circ g\) is an \emph{interior operator} (this is the terminology used in the
  warm\-/up). This is the preorder\-/variant of a more general property of
  adjunctions. Given any adjunction \(F \dashv G\) between categories, \(GF\) is
  a monad, and \(FG\) is a comonad.
\end{warning_box*}
\begin{proof}[Proof of Theorem~\ref{thm:galois-connection-closure}]
  The forward direction was proven in the above discussion. For the reverse
  direction, assume, for any \(x \in X\) and \(y \in Y\), \(x \leq g(f(x))\) and
  \(f(g(y)) \leq y\). We need to show that \(f\) and \(g\) form a Galois
  connection \(f \dashv g\). Consider a pair of elements \(x \in X\) and \(y \in
  Y\). We have a chain of implications.
  \begin{prooftree}
    \AxiomC{$x \leq g(f(x))$}
    \AxiomC{$f(x) \leq y$}
    \RightLabel{$g$ order\-/pres.}
    \UnaryInfC{$g(f(x)) \leq g(y)$}
    \RightLabel{$g\circ f$ closure operator.}
    \BinaryInfC{$x \leq g(y)$}
  \end{prooftree}
  Therefore \(f(x) \leq y\) implies \(x \leq g(y)\). Conversely, we have a
  similar chain of implications.
  \begin{prooftree}
    \AxiomC{$x \leq g(y)$}
    \LeftLabel{$f$ order\-/pres.}
    \UnaryInfC{$f(x) \leq f(g(y))$}
    \AxiomC{$f(g(y)) \leq y$}
    \LeftLabel{$f\circ g$ interior operator.}
    \BinaryInfC{$f(x) \leq y$}
  \end{prooftree}
  Thus \(x \leq g(y)\) implies \(f(x) \leq y\). Since \(x\) and \(y\) are
  arbitrary, this shows that \(f\) and \(g\) form a Galois connection \(f \dashv
  g\), as desired.
\end{proof}
\begin{example}
  During the seminar on Galois connections, Steven brought up a nice example
  for finding a Galois connection. Here is the example. It will motivate a
  theorem characterizing the existence of Galois connections. Consider the
  following preorders
  \[%
    \begin{tikzcd}
      X \eqdef & \bullet \arrow[r] & \bullet \arrow[r] & \bullet \arrow[r] & \bullet\\
      Y \eqdef & \bullet \arrow[r] & \bullet \arrow[r] & \bullet
    \end{tikzcd}
  \]%
  Given a functor \(g : Y \to X\), e.g.,
  \[%
    \begin{tikzcd}
      \bullet \arrow[r] & \bullet \arrow[r] & \bullet \arrow[r] & \bullet\\
      \bullet \arrow[u,bend right=10] \arrow[r] & \bullet \arrow[ur, bend
      right=10] \arrow[r] & \bullet
      \arrow[u, bend right=10]
    \end{tikzcd}
  \]%
  we can ask whether this functor has a left adjoint \(f: X \to Y\) (forming a
  Galois connection \(f \dashv g\). In this example, no left adjoint exists. We
  can see this by looking at what \(f\) must do to the right-most point of
  \(X\). Recall, for any Galois connection \(f \dashv g\) and any \(x \in X\),
  \(x \leq g(f(x))\). There is no choice for \(f(x)\) to satisfy this
  constraint, since the image of \(g\) is strictly less than the right-most
  point.

  On the other hand, we can ask if \(g : Y \to X\) has a \emph{right} adjoint
  \(h : X \to Y\) (forming a Galois connection \(g \dashv h\). We can construct
  \(h\) by looking at the behavior of \(g \circ h\) starting from the
  left. Recall, for a Galois connection \(g \dashv h\) and any element
  \(x \in X\), \(g(h(x)) \leq x\). For the left most point of \(X\), there is
  only one choice of \(h(x)\) which satisfies this constraint, shown below.
  \[%
    \begin{tikzcd}
      \bullet \arrow[r] \arrow[d,bend right=15, dashed] & \bullet \arrow[r] &
      \bullet \arrow[r] & \bullet\\
      \bullet \arrow[u,bend right=10] \arrow[r] & \bullet \arrow[ur, bend
      right=10] \arrow[r] & \bullet \arrow[u, bend right=10]
    \end{tikzcd}
  \]%
  Similarly, the image of the second point under \(h\) is fixed by this
  constraint.
  \[%
    \begin{tikzcd}
      \bullet \arrow[r] \arrow[d,bend right=15, dashed] & \bullet \arrow[r]
      \arrow[dl,bend right=10, dashed] & \bullet \arrow[r] & \bullet\\
      \bullet \arrow[u,bend right=10] \arrow[r] & \bullet \arrow[ur, bend
      right=10] \arrow[r] & \bullet \arrow[u, bend right=10]
    \end{tikzcd}
  \]%
  For the third point, any choice of image satisfies this constraint.
  \[%
    \begin{tikzcd}
      \bullet \arrow[r] \arrow[d,bend right=15, dashed] & \bullet \arrow[r]
      \arrow[dl,bend right=10, dashed] & \bullet \arrow[r] \arrow[dl,dashed,bend
      right=10, red] \arrow[dll,dashed,bend right=10, red] \arrow[d,dashed,bend
      right=15, red] & \bullet\\
      \bullet \arrow[u,bend right=10] \arrow[r] & \bullet \arrow[ur, bend
      right=10] \arrow[r] & \bullet \arrow[u, bend right=10]
    \end{tikzcd}
  \]%
  To determine which choice is the correct choice, we need to use another
  constraint. In the reverse direction, for any \(y \in Y\), \(y \leq
  h(g(y))\). When \(y\) is the right\-/most point of \(Y\), this fixes the
  choice of red arrow.
  \[%
    \begin{tikzcd}
      \bullet \arrow[r] \arrow[d,bend right=15, dashed] & \bullet \arrow[r]
      \arrow[dl,bend right=10, dashed] & \bullet \arrow[r] \arrow[d,dashed,bend
      right=15, red] & \bullet\\
      \bullet \arrow[u,bend right=10] \arrow[r] & \bullet \arrow[ur, bend
      right=10] \arrow[r] & \bullet \arrow[u, bend right=10]
    \end{tikzcd}
  \]%
  This leaves the right-most point of \(X\). \(h: X \to Y\) must be
  order\-/preserving. This fixes the choice for the image of the right\-/most
  point.
  \[%
    \begin{tikzcd}
      \bullet \arrow[r] \arrow[d,bend right=15, dashed] & \bullet \arrow[r]
      \arrow[dl,bend right=10, dashed] & \bullet \arrow[r] \arrow[d,dashed,bend
      right=15] & \bullet \arrow[dl,dashed,bend right=10, red]\\
      \bullet \arrow[u,bend right=10] \arrow[r] & \bullet \arrow[ur, bend
      right=10] \arrow[r] & \bullet \arrow[u, bend right=10]
    \end{tikzcd}
  \]%
  Thus, we have found the right adjoint \(h\) using the constraints \(g(h(x))
  \leq x\) and \(y \leq h(g(y))\).
\end{example}
The previous example suggests the question of uniqueness of an adjoint. Adjoints
are always unique, up to isomorphism. We will prove this by writing right
adjoints as colimits (and left adjoints as a limit).
\begin{theorem}\label{thm:right-adjoint-pointwise}
  Let \(f: X \to Y\) be an order\-/preserving map between preorders. If \(f\)
  has a right adjoint \(g: Y \to X\), then \(g\) satisfies
  \[%
    g(y) \cong \bigvee_{x \in f^{-1}(\dset y)} x.
  \]
\end{theorem}
\begin{warning_box*}
  This supremum is the preorder variant of the following colimit, which
  characterizes a right adjoint of a functor \(F: C \to D\).
  \[%
    G(y) \cong \varinjlim\pwrap{ F \downarrow y \to C },
  \]%
  where \(F \downarrow y\) is the \emph{comma category} of \(F\) and \(y\). Its
  objects are pairs \(\pwrap{x, a: F(x) \to y}\) of an object \(x\) of \(C\) and
  an arrow \(a : F(x) \to y\) in \(D\). An arrow \(\pwrap{x,a} \to
  \pwrap{x',a'}\) is an arrow \(\theta: x \to x'\) in \(C\) fitting into a
  commuting triangle
  \[%
    \begin{tikzcd}
      F(x) \arrow[rr,"F(\theta)"] \arrow[dr,"a",bend right=10] & & F(x')
      \arrow[dl,"a'", bend left=10, swap] \\
      & y. &
    \end{tikzcd}
  \]%
  The functor \(F \downarrow y \to C\) sends a commuting triangle likke the one
  above to the arrow \(\theta : x \to x'\).
\end{warning_box*}
\begin{proof}[Proof of Theorem~\ref{thm:right-adjoint-pointwise}]
  We first show that \(g(y)\) is an upper bound for \(f^{-1}(\dset y)\). Then we
  will show that \(g(y)\) is the least upper bound. Any element
  \(x \in f^{-1}(\dset y)\) satisfies \(f(x) \leq y\) by definition of
  \(\dset y\). Since \(g\) is a right adjoint, this implies \(x \leq g(y)\), so
  \(g(y)\) is an upper bound. Furthermore, any Galois connection \(f \dashv g\)
  satisfies
  \[%
    f(g(y)) \leq y,
  \]%
  for any \(y \in Y\). Thus \(g(y) \in f^{-1}(\dset y)\), so \(g(y)\) is a least
  upper bound, as desired. (In fact, this shows that \(g(y)\) is the maximum of
  \(f^{-1}(\dset y)\) rather than just the supremum.)
\end{proof}
\begin{corollary}
  Let \(f: X \to Y\) be an order\-/preserving map between preorders. Suppose
  \(g_1 : Y \to X\) and \(g_2: Y \to X\) are right adjoints to \(f\), i.e.,
  there are Galois connections \(f \dashv g_1\) and \(f \dashv g_2\). Then
  \(g_1\) and \(g_2\) are \textbf{naturally isomorphic}: for any element \(y \in
  Y\),
  \[%
    g_1(y) \leq g_2(y) \text{ and } g_2(y) \leq g_1(y).
  \]%
\end{corollary}
\begin{proof}
  By Theorem~\ref{thm:right-adjoint-pointwise}, the two Galois connections \(f
  \dashv g_1\) and \(f \dashv g_2\) imply, for any \(y \in Y\)
  \[%
    g_1(y) \cong \bigvee_{x \in f^{-1}(\dset y)} x \cong g_2(y).
  \]%
  Thus \(g_1(y) \cong g_2(y)\). Recall this means \(g_1(y) \leq g_2(y)\) and
  \(g_2(y) \leq g_1(y)\), as desired.
\end{proof}
We can prove the analogous things for left adjoints too.
\begin{warning_box*}
  \begin{exercise}
    Let \(g: Y \to X\) be an order\-/preserving map between preorders. Suppose
    \(f: X \to Y\) is a \emph{left} adjoint to \(g\). Show, for any \(x \in X\),
    \[%
      f(x) \cong \bigwedge_{y \in g^{-1}(\uparrow x)} y.
    \]%
    Where \(\uparrow x\) is the set of elements \(x'\) in \(X\) such that \(x \leq
    x'\). This is the preorder form of the limit
    \[%
      f(x) \cong \varprojlim{x \downarrow g \to Y}.
    \]%
    Furthermore, show that left adjoints are unique up to natural isomorphism.
    
    This exercise can be proven directly, using the same argument. Alternatively,
    use Theorem~\ref{thm:right-adjoint-pointwise} on the opposite preorders
    \(\pwrap{X,\geq}\) and \(\pwrap{Y,\geq}\).
  \end{exercise}
\end{warning_box*}

\section{A very important Galois connection}

Let \(X\) and \(Y\) be a pair of sets, and let \(f: X \to Y\) be a function
between them. From \(f\) we can define an order\-/preserving map
\(f^{-1} : \Pcal(Y) \to \Pcal(X)\) which sends a subset \(A \in \Pcal(Y)\) to
its preimage
\[%
  f^{-1}(A) \eqdef \bwrap{ x \in X \mid f(x) \in A } \in \Pcal(X).
\]%
This order\-/preserving map participates in two Galois connections. These Galois
connections are essential for interpreting first\-/order logic inside a
topos. First, \(f^{-1}\) has a left adjoint, written
\(\exists_f : \Pcal(X) \to \Pcal(Y)\). This is the usual image map:
\[%
  \exists_f \pwrap{B \in \Pcal(X)} \eqdef f(B) = \bwrap{ y \in Y \mid \exists x
    \in X. f(x) = y \wedge x \in B }.
\]%
\begin{proposition}
  \(\exists_f\) is the left adjoint of \(f^{-1}\) (forming a Galois connection
  \(\exists_f \dashv f^{-1}\)).
\end{proposition}
\begin{proof}
  Let \(A \in \Pcal(Y)\) and \(B \in \Pcal(X)\). Then \(B \subseteq f^{-1}(A)\)
  if and only if every element \(x \in B\) satisfies \(f(x) \in A\), which is
  true if and only if \(f(B) \subseteq A\), i.e., \(\exists_f B \subseteq A\).
\end{proof}
\(f^{-1}\) also has a right adjoint \(\forall_f : \Pcal(X) \to \Pcal(Y)\). This
is sometimes called the \emph{dual image} (c.f.\linebreak\cite{Makkai1977}). The
dual image is defined using the universal quantifier:
\[%
  \forall_f \pwrap{B \in \Pcal(X)} \eqdef \bwrap{%
    y \in Y \mid \forall x \in X. f(x) = y \Rightarrow x \in B
  } = \bwrap{%
    y \in Y \mid f^{-1}(\bwrap{y}) \subseteq B
  }.%
\]%
\begin{proposition}
  \(\forall_f\) is a right adjoint for \(f^{-1}\), yielding a Galois connection
  \(f^{-1} \dashv \forall_f\).
\end{proposition}
\begin{proof}
  Let \(A \in \Pcal(Y)\) and \(B \in \Pcal(X)\) be a pair of subsets. Suppose
  \(f^{-1}(A) \subseteq B\). Then for any element \(y \in A\),
  \(f^{-1}(\bwrap{y}) \subseteq B\) (since \(f^{-1}\) is order\-/preserving),
  i.e., \(y \in \forall_f A\). Since \(y\) is arbitrary, this shows that \(A
  \subseteq \forall_f B\). Conversely, suppose \(A \subseteq \forall_f B\), then
  for any \(y \in A\), \(f^{-1}(\bwrap{y}) \subseteq B\). Thus
  \[%
    f^{-1}(A) = \bigcup_{y \in A} f^{-1}(\bwrap{y}) \subseteq B,
  \]%
  as desired.
\end{proof}
The previous two propositions show that we have an \emph{adjoint triple}
\(\exists_f \dashv f^{-1} \dashv \forall_f\). Adjoint triples are particularly
nice categorical structures. Longer adjoint \emph{strings}, e.g., \(f_1 \dashv
\ldots \dashv f_n\) are rarer and have increasingly powerful properties. For
example, we have a fun fact.
\begin{proposition}[\cite{Rosebrugh1994}]
  Let \(C\) be a locally small category, and let \(\Yo : C \to \text{Psh}(C)\)
  be its Yoneda embedding. If \(\Yo\) extends to the left to an adjoint
  quintuple
  \[%
    F_1 \dashv F_2 \dashv F_3 \dashv F_4 \dashv \Yo,
  \]%
  then \(C\) is equivalent to the category \(\Set\).
\end{proposition}
The adjoint triple \(\exists_f \dashv f^{-1} \dashv \forall_f\) gives an
overpowered proof of the following result.
\begin{corollary}
  \(f^{-1} : \Pcal(Y) \to \Pcal(X)\) preserves intersections and unions.
\end{corollary}
\begin{proof}
  \(f^{-1}\) has both a left and a right adjoint. Therefore, by
  Proposition~\ref{prop:adjoints-preserve-limits}, \(f^{-1}\) preserves infima
  and suprema, i.e., intersections and unions.
\end{proof}

\subsection*{Quantifiers as Adjoints}

The notation in the adjoint triple \(\exists_f \dashv f^{-1} \dashv \forall_f\)
suggests the adjoints to \(f^{-1}\) have something to say about the quantifiers
in first\-/order predicate logic. If we are working within a topos, there is a
bijection between subsets \(A \subseteq X\) and propositions \(\chi_A(x : X)\)
with a free variable representing an element of \(X\). This bijection is given
by
\[%
  \phi(x : X) \mapsto \bwrap{x \in X \mid \phi(x)} \subseteq X.
\]%
(This bijection exists because a topos has a \emph{subobject classifier}.) Under
this bijection, the above adjoint triple interprets the quantifiers \(\exists\)
and \(\forall\) in the way we expect. We can also describe essentially the same
adjoint triple directly in logic.

Recall, from a propositional theory \(T\), we can form its Lindenbaum algebra
\(L(T)\) of propositions. To describe a first\-/order theory, we stitch together
families of Lindenbaum algebras, indexed by a context of free variables.
\begin{definition}
  A \textbf{first\-/order signature} \(\Sigma\) is a collection of the following
  data.
  \begin{itemize}
  \item A set of \emph{sorts} \(\sigma \in \Sigma\).
  \item A collection of base terms, i.e., \emph{function symbols}
    \[%
      x_1 : \sigma_1, \ldots, x_n : \sigma_n \vdash f(x_1,\ldots,x_n) : \tau.
    \]%
  \item For every context \(x_1: \sigma_1,\ldots,x_n : \sigma_n\), a collection
    of base propositions. e.g., \(R(x_1 : \sigma_1,\ldots, x_n : \sigma_n)\).
  \end{itemize}
  A \textbf{first\-/order theory} is a first\-/order signature and a set of
  sequents, which are the axioms of the theory.
\end{definition}
If we fix a context, we can use the usual connectives \(\wedge,\vee,\) etc.\ to
form formulae. The new expressivity comes from the terms in a first\-/order
signature. We can freely compose terms in the usual typed manner. Given a list
of terms \(\Gamma \vdash t_1 : \tau_1, \ldots, \Gamma \vdash t_n : \tau_n\),
which we could write as an arrow \(t: \Gamma \to \Gamma'\), where \(\Gamma'\) is
a context of fresh variables of type \(\tau_1,\ldots,\tau_n\), we obtain a map
\[%
  t^* : \text{Form}(\Gamma') \to \text{Form}(\Gamma)
\]%
given by substitution. If \(\phi( \vec{x} : \Gamma' )\) is a formula with
context \(\Gamma'\), then substituting the list of terms \(t\) gives a formula
\(\phi(t_1,\ldots,t_n)\) with context \(\Gamma\). The left and right adjoints to
\(t^*\) allow us to interpret quantifiers: \(\exists_t \dashv t^* \dashv
\forall_t\), where the preorder on \(\text{Form}(\Gamma)\) is given by
provability \(\vdash\), the same as in the Lindenbaum algebra. We bundle this
data into one categorical structure.
\begin{definition}
  A \textbf{first\-/order hyperdoctrine} is a pair \(\pwrap{C,F}\) of a category
  \(C\) of contexts and terms and a functor
  \[%
    F : C^\op \to \text{Lindenbaum Algebras}
  \]%
  sending a context \(\Gamma\) to its Lindenbaum algebra of formulae with free
  variables in \(\Gamma\). We require, for any arrow \(t : \Gamma \to \Gamma'\)
  in \(C\), we have an adjoint triple
  \[%
    \exists_t \dashv F(t) \dashv \forall_t.
  \]%
\end{definition}
Any first\-/order theory defines a first\-/order hyperdoctrine, where \(F(t) =
t^*\). The adjoints \(\exists_t\) and \(\forall_t\) are given by
\[%
  \exists_t \phi( \vec{x'} : \Gamma' ) \equiv \exists \vec{x} : \Gamma
  .\, t(\vec{x}) = \vec{x'} \wedge \phi(\vec{x}).
\]%
and
\[%
  \forall_t \phi(\vec{x'} : \Gamma' ) \equiv \forall \vec{x} : \Gamma.\,
  t(\vec{x}) = \vec{x'} \Rightarrow \phi(\vec{x}).
\]%
\begin{theorem}[Quantifiers are adjoints]
  Given a pair of context \(\Gamma\) and \(\Gamma'\), a list of terms \(t :
  \Gamma \to \Gamma'\), and a pair of formulae \(\phi( \vec{x} : \Gamma )\) and
  \(\psi(\vec{x'} : \Gamma')\),
  \[%
    \phi(\vec{x}) \vdash t^* \psi(\vec{x}) \text{ if and only if } \exists_t
    \phi(\vec{x'}) \vdash \psi(\vec{x'}).
  \]%
  Similarly
  \[%
    t^* \psi(\vec{x}) \vdash \phi(\vec{x}) \text{ if and only if } \forall_t
    \phi(\vec{x'}) \vdash \psi(\vec{x'}).
  \]%
  In other words, we have a pair of Galois connections \(\exists_t \dashv t^*\)
  and \(t^* \dashv \forall_t\).
\end{theorem}

We now give an example of a term and its adjoint triple. Given a pair of
variables \(x_1 : \sigma_1, x_2 : \sigma_2\), we can form the ``projection''
\[%
  x_1 : \sigma_1, x_2: \sigma_2 \vdash x_1 : \sigma_1,
\]%
representing an arrow \(\pi_1 : \sigma_1,\sigma_2 \to \sigma_1\). Let
\(\phi(x_1,x_2\) and \(\psi(x_1\) be a pair of formulae with context
\(\sigma_1,\sigma_2\) and \(\sigma_1\), respectively. Note, given another pair
of variables \(y_1 : \sigma_1, y_2 : \sigma_2\), then
\[%
  \pi_1^* \psi(y_1,y_2) \equiv \psi(\pi_1(y_1,y_2)) \equiv \psi(y_1),
\]%
Similarly,
\[%
  \exists_{\pi_1} \phi(x_1) \equiv \exists x_1' : \sigma_1.\, \exists x_2' :
  \sigma_2 .\, \pi_1(x_1',x_2') = x_1 \wedge \phi(x_1',x_2') \equiv \exists x_1'
  \exists x_2' \, x_1' = x_1 \wedge \phi(x_1',x_2').
\]%
The latter is provably equivalent (by \(=\) elimination) to \(\exists x_2'
.\, \phi(x_1, x_2')\). To emphasize, we have shown:
\[%
  \exists_{\pi_1} \phi(x_1) \dashv \vdash \exists x_2' .\, \phi(x_1,x_2').
\]%
A similar argument shows \(\forall_{\pi_1} \phi(x_1) \dashv\vdash \forall x_2'
.\, \phi(x_1,x_2')\). In other words, the adjoints \(\exists_\pi\) and
\(\forall_\pi\) recover the usual quantifiers when \(\pi\) is a projection.

\bibliographystyle{apalike}
\bibliography{../refs}

\end{document}
