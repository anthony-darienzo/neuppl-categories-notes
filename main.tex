\documentclass{./thesis-note}

\usepackage{bussproofs}

\definecolor{NortheasternRed}{RGB}{200,16,46}
\titlecolor{NortheasternRed}

\subtitle{Categorical Semantics of Programming Languages}
\title{\color{NortheasternRed} NeuPPL Categories Seminar}
\author{Anthony D'Arienzo}
\contact{apd6@illinois.edu}
\date{Summer 2024}

\providecommand\natsubseteq{{\stackrel{\text{nat}}{\leq}}}
\newcommand\Prop[1]{\text{{\normalfont{Prop}}}{\pwrap{{#1}}}}
\DeclareMathOperator\Mod{\normalfont Mod}

\newcommand\Dwd{\text{\normalfont Dwd}}
\newcommand\natleq{\mathrel{\stackrel{\text{nat}}{\leq}}}
\DeclareMathOperator\dset{\downarrow}

\newcommand\Pcal{{\mathcal{P}}}

\DeclareMathOperator\el{el}
\newcommand\iSet[1]{{{#1}\text{\normalfont{-Set}}}}

\tikzset{snake it/.style={decorate, decoration=snake}}

\usepackage{subfiles}

\begin{document}

\doublespacing%
\vspace{\fill}
\makehomeworktitle%
\onehalfspacing%
\vspace{0.05\paperheight}

These notes are lecture notes from a summer seminar at
Northeastern University to study category theory and its applications
to the semantics of programming languages. The first half of these
notes were written during the seminar, whereas the latter half was
written after the last meeting. The contents of the latter half are,
more or less, what was covered in these meetings.

Our goal with this seminar was to develop the category theory
necessary to understand the relationship between step\-/indexed
logical relations and the Kripke\-/Joyal semantics of the topos of
trees, i.e., the topos of presheaves over the first countable
ordinal. The seminar concluded with an overview of Birkedal et al's
article \emph{First Steps in Synthetic Guarded Domain Theory:
  Step\-/Indexing in the Topos of Trees} \cite{5970227}.

In between meetings, we read Fong and Spivak's \emph{An Invitation to
  Applied Category Theory}\linebreak\cite{Fong2019}. As such, these
notes assume that the reader has an idea for what is a category and a
functor---though perhaps not familiar with category
\emph{theory}!---approximately at the level of someone who has seen
the Curry\-/Howard correspondence between certain categories and the
simply typed \(\lambda\) calculus. On the other hand, these notes, and
our seminar, were designed as a review and crash\-/course of the
aspects of category theory needed to describe the Kripke\-/Joyal
semantics of a topos. This aim motivates the order in which things are
presented. First, the connection to propositional logic and preordered
sets is established. We spend an extended time with preordered sets in
order to develop the key results of category theory for this special
case. We then introduce categories as an intensional/proof-relevant
generalization of a preorder and presheaves as the corresponding
generalization of a downward-closed set.

This preorder\-/focused approach gives these notes a unique
perspective on category theory. It is my hope that they emphasize the
role of a topos as a context\-/aware universe of sets. These
context\-/aware ``sets'' appear in the logical relations of virtually
any programming language with more features than the simply\-/typed
\(\lambda\) calculus, e.g., languages with recursive types or with
references. These logical references can be complex: any quantifiers
involved in their definition are guarded with conditions described
their dependence on contextual data. In the case of recursive types,
these extra conditions involve a \emph{step\-/index}. In the case of
references, these conditions involve \emph{heap extensions}. In any
case, these conditions are precisely those which appear from the
Kripke\-/Joyal semantics. This gives the PL theorist a new tool: by
working internal to a topos of contexts, one can replace the indexed
logical relations and their extra conditions with the simpler
relations used for the simply typed \(\lambda\) calculus.

A special thanks to Steven Holtzen for his hosting me this summer at
Northeastern as well as his lab for welcoming me.
\vspace{\fill}

\subtitle{}
\title{}
\author{}
\contact{}
\date{}

\doublespacing
\subfileinclude{contents/contents}
\onehalfspacing
\subfileinclude{notes1/notes1}
\subfileinclude{hw0/hw0}
\subfileinclude{notes2/notes2}
\subfileinclude{notes3/notes3}
\subfileinclude{notes4/notes4}

\bibliographystyle{apalike}
\bibliography{./refs}

\end{document}
%%% Local Variables:
%%% mode: latex
%%% TeX-master: t
%%% End: