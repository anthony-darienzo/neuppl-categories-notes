\documentclass[../main.tex]{subfiles}

\begin{document}

\ifSubfilesClassLoaded{%
    \title{Category Theory on Preorders}
    \subtitle{}
    \author{}
    \contact{Summer 2024}
    \date{NeuPPL Category Seminar}}{%
    \title{Category Theory on Preorders}
    \subtitle{}
    \author{}
    \contact{}
    \date{}}

\makehmtitle%

\section{Preorders are Indexed Propositions}

Let \(\pwrap{X,\leq}\) be a preorder. Then we can form the classifying category
\(\Cl\pwrap{X,\leq}\) as before. We also defined
\(\Omega \eqdef \bwrap{ \True , \False }\). This is the preorder of
\emph{propositions}. It is essentialy the preorder of booleans
\texttt{bool}. However, for foundational reasons, it is useful to think of
\(\Omega\) has having slightly less structure than the full boolean
operations. In the first warmup, we constructed a functor
\(\dset x : \Cl\pwrap{X,\leq}^\op \to \Omega\) by forming the
\emph{downset}
\[%
  \dset x (y) \eqdef
  \begin{cases}
    \True & \text{if \(y \leq x\)}\\
    \False & \text{otherwise.}
  \end{cases}
\]%
The downset construction \(x \mapsto \dset x\) allows us to reason about
the abstract preorder \(\pwrap{X,\leq}\) as if it were a collection of
propositions, indexed by the elements of \(X\).
\begin{definition}
  Given a preorder \(\pwrap{X,\leq}\), a \textbf{downward closed set} is a
  subset \(A \subseteq X\) satisfying the following closure property:
  \[%
    \begin{array}{c}
      x \leq y \\\hline
      y \in A \Rightarrow x \in A.
    \end{array}
  \]%
\end{definition}
There is another way to represent downward closed sets, in terms of functions
\(X \to \Omega\). Given a subset \(A \subseteq X\), we can form the
\emph{membership function} \(\chi_A : X \to \Omega\):
\[%
  \chi_A(x) \eqdef
  \begin{cases}
    \True & x \in A,\\
    \False & \text{otherwise}.
  \end{cases}
\]%
Conversely, any function \(\varphi: X \to \Omega\) defines a subset
\(\varphi^{-1}(\True) \eqdef \bwrap{ x \in X \mid \varphi(x) = \True}\).
\begin{proposition}
  A subset \(A \subseteq X\) is downward closed if and only if its membership
  function \(\chi_A : X \to \Omega\) is order\-/reversing. Similarly, a function
  \(\varphi: X \to \Omega\) is order\-/reversing if and only if its
  corresponding subset \(\varphi^{-1}\pwrap{\True}\) is downward closed.
\end{proposition}
\begin{proof}
  Consider a subset \(A \subseteq X\). Suppose that \(A\) is downward closed,
  i.e.,
  \[%
    \begin{array}{c}
      x \leq y \\\hline
      y \in A \Rightarrow x \in A.
    \end{array}
  \]%
  Note \(y \in A\) if and only if \(\chi_A(y) = \True\). Therefore the above
  judgment is equivalent to
  \[%
    \begin{array}{c}
      x \leq y \\\hline
      \chi_A(y) = \True \Rightarrow \chi_A(x) = \True.
    \end{array}
  \]%
  This is the statement that \(\chi_A\) is order\-/reversing. Conversely, if
  \(\chi_A\) is assumed to be order\-/reversing, then the equivalence of the
  above two judgments implies \(A\) is downward closed.

  The result for functions \(\varphi: X \to \Omega\) follows a similar
  argument.
\end{proof}
The above proposition demonstrates a bijection between the following sets of
structures
\[%
  \begin{array}{ccc}
    \bwrap{ \text{Downward\-/closed subsets \(A \subseteq X\)} } &%
    \cong &%
    \bwrap{ \text{Order\-/reversing functions \(X \to \Omega\)} }\\
    A & \mapsto & \chi_A \\
    \varphi^{-1}\pwrap\True & \mapsfrom & \varphi
  \end{array}
\]%
Recall that an order\-/preserving function \(\pwrap{X,\leq} \to \pwrap{Y,\leq}\)
is equialent to the data of a functor between classifying categories
\(\Cl\pwrap{X,\leq} \to \Cl\pwrap{Y,\leq}\). Furthermore, an order\-/reversing
function \(\pwrap{X,\leq} \to \pwrap{Y,\leq}\) is the same data as an
order\-/\emph{preserving} function---provided the preorder on \(X\) is replaced
with the reverse preorder \(\geq\). In other words, we have a chain of
bijections
\begin{align*}%
  \bwrap{ \text{Order\-/reversing functions
      \(\pwrap{X,\leq} \to \pwrap{\Omega,\leq}\)} } &\cong \bwrap{
    \text{Order\-/preserving functions \(\pwrap{X,\geq} \to
      \pwrap{\Omega,\leq}\)} }\\
  &\cong \bwrap{ \text{Functors \(\Cl\pwrap{X,\leq}^\op \to \Omega\)} }.
\end{align*}%
Combining this with the preceding proposition on downward closed sets, we have
obtained the following theorem.
\begin{theorem}[Grothendieck construction for preorders]
  There is a bijective correspondence
  \[%
    \bwrap{\text{Downward-closed subsets \(A \subseteq X\)}} \cong
    \bwrap{\text{Functors \(\Cl\pwrap{X,\leq}^\op \to \Omega\)}}
  \]%
  given by sending a downward closed subset \(A\) to the functor associated to
  \(\chi_A\) and by sending a functor \(F : \Cl\pwrap{X,\leq}^\op \to \Omega\)
  to the set of objects \(x \in X\) such that \(F(x) = \True\).
\end{theorem}
\begin{warning_box}{Remark}
  The above theorem allows us to make our first abuse of notation! In these
  notes we will now identify the functor \(\dset x : \Cl\pwrap{X,\leq}^\op \to
  \Omega\) with its associated downward closed set
  \[%
    \bwrap{ y \in X \mid \dset x(y) = \True }.
  \]%
  In retrospect, it would have been better to work entirely with
  downward\-/closed sets first, and introducing functors later. In the sequel,
  where we repeat the above with presheaves, it will be cleaner to avoid this
  abuse of notation.
\end{warning_box}
Downward closed sets inherit a partial order given by inclusion of subsets. This
allows us to talk about the \textbf{partial order of downward closed sets}
\[%
  \pwrap{\Dwd(X), \subseteq}.
\]%
The downset construction \(x \mapsto \dset x\) is an embedding \(X \to
\Dwd(X)\) with some nice properties.
\begin{lemma}[Yoneda lemma for preorders]
  The operation \(\dset : X \to \Dwd(X)\) is fully faithful, i.e.,
  \(x \leq y\) if and only if \(\dset x \subseteq \dset y\). In
  particular, \(\dset x = \dset y\) if and only if \(x \leq y\) and
  \(y \leq x\). Furthermore, for any downward closed set \(A \in \Dwd(X)\),
  \[%
    x \in A \text{ if and only if } \dset x \subseteq A.
  \]%
\end{lemma}
\begin{proof}
  The first part, that \(\dset\) is fully faithful, was Exercise 3 of the
  warmup. The second part, that \(x \in A\) if and only if \(\dset x \subseteq
  A\) follows a similar proof.

  Assume \(x \in A\). The downset associated to \(\dset x\) is the collection of
  all elements \(y \in X\) such that \(y \leq x\). Thus,
  \begin{prooftree}
    \AxiomC{$x \in A$}
    \AxiomC{$y \in \dset x$}
    \RightLabel{Defn. $\dset x$}
    \UnaryInfC{$y \leq x$}
    \RightLabel{$A \in \Dwd(X)$}
    \BinaryInfC{$y \in A$.}
  \end{prooftree}
  Since \(y\) is arbitrary, this shows \(\dset x \subseteq A\), as
  desired. Conversely, assume \(\dset x \subseteq A\). By reflexivity, \(x \leq
  x\), so \(x \in \dset x\). In particular, \(x \in A\), as desired.
\end{proof}
\begin{remark}
  Exercise 3 and the above lemma are proven using the same technique. The
  judgment \(\dset x \subseteq A\) relates arrows \(y \to x\) to elements of
  \(A\). Among all arrows \(y \to x\), there is a unique ``universal'' choice,
  namely the identity \(\id_x: x \to x\) (universal in the sense that it always
  exists). In similar situations, where one is quantifying over arrows in a
  category, picking the identity is often a decisive trick.
\end{remark}
The Grothendieck construction for preorders and the Yoneda lemma allow us to
study a preorder \(\pwrap{X,\leq}\) through its downward closed sets
\(\pwrap{\Dwd(X),\subseteq}\). In this sense, the only preorder relation is the
subset inclusion relation \(\subseteq\): all others are abstracted
``substructures'' of a preorder of subsets, if we pay the price of treating
a pair of elements \(x,y \in X\) as ``equivalent'' if \(x \leq y\) and \(y \leq
x\). This price is often worth paying, since \(\Dwd(X)\) has more structure than
\(X\).
\begin{observation}
  \(\Dwd(X)\) is a complete Heyting algebra.
\end{observation}
Since \(\Dwd(X)\) is a complete Heyting algebra, we can freely take unions and
intersections of downward-closed sets. This will be the basis of the following
section.

\section{Limits and Colimits of Preorders}

In the previous section, we saw that we can study a preorder \(\pwrap{X,\leq}\)
using its collection of downward closed sets \(\Dwd(X)\). This was justified by
the embedding \(\dset : X \to \Dwd(X)\). Furthermore, \(\Dwd(X)\) is a complete
Heyting algebra. We can use this Heyting algebra structure to define operations
in \(X\). For example, we can form intersections.
\begin{definition}
  Let \(x,y \in X\) be a pair of elements. A \textbf{conjunction} of \(x,y\) is
  an element \(z \in X\) which satisfies the equation of downward-closed
  subsets.
  \[%
    \dset z = \dset x \cap \dset y.
  \]%
\end{definition}
In other words, a conjunction in \(X\) of \(x\) and \(y\) is an element which
behaves like the intersection of \(x\) and \(y\). Since the embedding \(\dset :
X \to \Dwd(X)\) is fully faithful, this can be formalized in the manner
above. Another way to describe a conjunction is via a \emph{universal
  property}.
\begin{proposition}\label{prop:conjunction}
  Let \(x,y \in X\) be a pair of elements. An element \(z \in X\) is a
  conjunction of \(x\) and \(y\) if and only if, for any element \(w \in X\),
  \[%
    w \leq z \text{ if and only if } w \leq x \text{ and } w \leq y.
  \]%
\end{proposition}
In category theory, we usual write universal properties like the one in the
above proposition as a diagram. In this case, the diagram depicts arrows in the
classifying category \(\Cl\pwrap{X,\leq}\). Here is the diagram for a binary
conjunction.
\[%
  \begin{tikzcd}
    w \arrow[dr,dashed,"\exists !"] \arrow[drr,bend left=30] \arrow[ddr,bend
    right=30] & & \\
    & z \arrow[r] \arrow[d] & x \\
    & y &
  \end{tikzcd}
\]%
Every arrow in this diagram, being an arrow in \(\Cl\pwrap{X,\leq}\) represents
a comparison using \(\leq\). The solid arrows represent the hypotheses of the
universal property, and the dashed line represents the conclusion. In this case,
the diagram translates to
\begin{indented}
  \centering
  \emph{Given \(z \leq x\) and \(z \leq y\). If \(w \leq x\) and \(w \leq y\),
    then \(w \leq z\).}
\end{indented}
I have also added the symbol \(\exists !\) to the diagram. This is not relevant
for preorders, because there is always at most one arrow between any two objects
of \(\Cl\pwrap{X,\leq}\). However, in general category theory, universal
properties almost always require the dashed line to be unique.
\begin{proof}[Proof of Proposition~\ref{prop:conjunction}]
  Assume \(z\) is a conjunction of \(x\) and \(y\). We need to show, for any
  element \(w \in X\), \(w \leq z\) if and only if \(w \leq x\) and \(w \leq
  y\). We now chase a chain of iffs
  \begin{align*}
    w \leq z &\text{ iff } w \in \dset z\\
    &\text{ iff } w \in \dset x \cap \dset y\\
    &\text{ iff } w \in \dset x \text{ and } w \in \dset y\\
    &\text{ iff } w \leq x \text{ and } w \leq y.
  \end{align*}
  Here the second iff holds because \(z\) is a conjunction of \(x\) and
  \(y\). The converse direction is a similar argument.
\end{proof}
\begin{remark}
  In the above, \(z\) is called \emph{a} conjunction of \(x\) and \(y\) rather
  than \emph{the} conjunction of \(x\) and \(y\). This is because conjunctions
  are technically not unique---but they are unique \emph{up to
    isomorphism}. Suppose \(z\) is a conjunction of \(x\) and \(y\), and suppose
  that there exists an element \(z' \in X\) such that \(z' \leq z\) and
  \(z \leq z'\). (These conditions are equivalent to the statement that \(z\)
  and \(z'\) are isomorphic in \(\Cl\pwrap{X,\leq}\).) Then \(\dset z' = \dset
  z\), so \(z'\) is also a conjunction of \(x\) and \(y\). In practice,
  identifying isomorphic objects is manageable, so we usually give the
  isomorphism class of conjunctions of \(x\) and \(y\) a name
  \[%
    z \cong x \wedge y.
  \]%
  Furthermore, it is common in category theory to write \(x \wedge y\) instead
  of a explicit representative of the conjunction.
\end{remark}
We now consider two examples.
\begin{example}[A preorder of subsets]
  Suppose \(\pwrap{X,\leq}\) is a preorder of subsets \(\pwrap{X,\leq} =
  \pwrap{\mathcal{P}(V),\subseteq}\). Consider a pair of subsets \(A, B \in
  \mathcal{P}(V)\). A conjunction \(A \wedge B\) of \(A\) and \(B\) satisfies
  the universal property:
  \[%
    \forall W \subseteq V, W \subseteq A \wedge B \Leftrightarrow W \subseteq A
    \text{ and } W \subseteq B.
  \]%
  If we pick singleton subsets \(\bwrap{v}\) for \(W\) (so \(v\) is some element
  of \(V\)), then the above universal property implies
  \[%
    \forall v \in V, v \in A \wedge B \Leftrightarrow v \in A \text{ and } v \in
    B.
  \]%
  This sentence uniquely characterizes the intersection \(A \cap B\). Therefore
  \[%
    A \wedge B \cong A \cap B.
  \]%
  (Furthermore, since \(\subseteq\) is an antisymmetric relation, this
  immediately implies \(A \wedge B = A \cap B\).) Power sets are the nicest type
  of preorder: they have a generating set of singleton sets, also called
  \emph{atoms}, which can distinguish any pair of objects in the power set. This
  property is called \emph{well\-/pointedness}. For this reason, we could be
  silly and say that a power set \(\mathcal{P}(V)\) is an example of a
  \emph{atomic 0\-/topos with enough points}. This probably doesn't help to
  understand preorders, but it can be useful to intuit how an atomic topos
  generalizes the power-set situation!
\end{example}
\begin{example}[A propositional theory]
  Let \(T = \pwrap{\Sigma,\Delta}\) be a propositional theory. We constructed
  the classifying category of \(T\), \(\Cl(T)\), whose objects are propositions
  \(\phi\) in \(T\) and whose arrows \(\phi \to \psi\) correspond to sequents
  \(\phi \vdash_T \psi\). Given two propositions \(\phi\) and \(\psi\) of \(T\),
  we can ask for their conjunction. This conjunction is a proposition, \(P\)
  such that, for any other proposition \(Q\),
  \[%
    Q \vdash_T P \Leftrightarrow Q \vdash_T \phi \text{ and } Q \vdash_T \psi.
  \]%
  The latter condition, \(Q \vdash_T \phi\) and \(Q \vdash_T \psi\) is
  equivalent to \(Q \vdash_T \phi \wedge \psi\). Thus, a conjunction \(P\)
  satisfies
  \[%
    Q \vdash_T P \Leftrightarrow Q \vdash_T \phi \wedge \psi.
  \]%
  This implies \(P \vdash_T \phi \wedge \psi\) and \(\phi \wedge \psi \vdash_T
  P\). Thus \(P \cong \phi \wedge \psi\). In other words, the conjunctions in
  \(\Cl\pwrap{T}\) are provably equivalent to the usual conjunctions from
  propositional logic. This observation justifies our use of the symbol
  \(\wedge\) to denote conjunction. It also an example of the motivation of
  categorical logic: \emph{features in a language correspond to universal
    properties of the classifying category}.
\end{example}
We defined conjunctions by an equation of downsets. Nothing is stopping us from
picking other equations in order to study other universal properties. For
example, we could ask for larger conjunctions. Given a family \(x_\alpha\) of
elements of \(X\), a conjunction \(\bigwedge_\alpha x_\alpha\) is an element \(z
\in X\) satisfying the equation
\[%
  \dset z = \bigcap_\alpha \dset x_\alpha.
\]%
From this we can derive the universal property like before. Instead we move
ahead to the general picture.
\begin{definition}
  Let \(\pwrap{X,\leq}\) be a preordered set. A \textbf{diagram} is an
  order\-/preserving map \(J : \pwrap{D,\leq} \to \pwrap{X,\leq}\). In other
  words, a diagram is a functor \(J : \Cl\pwrap{D,\leq} \to
  \Cl\pwrap{X,\leq}\).
\end{definition}
To every diagram \(J : \pwrap{D,\leq} \to \pwrap{X,\leq}\), we can create a
downward closed set \(\dset J\) in \(\Dwd(X)\):
\[%
  \dset J \eqdef \bigcap_{i \in D} \dset \pwrap{J(i)}.
\]%
Thus, \(x \in \dset J\) if and only if \(x \leq J(i)\) for every \(i \in D\).
\begin{definition}
  Given a diagram \(J : \pwrap{D,\leq} \to \pwrap{X,\leq}\), a \textbf{limit} of
  \(J\) is an element \(z \in X\) such that
  \[%
    \dset z = \dset J
  \]%
  as elements of \(\Dwd(X)\). Similar to how conjunctions were written using
  \(\wedge\), e.g., ``\(x \wedge y\)'', we write for limits
  \[%
    z \cong \varprojlim_{i \in D} J(i) \text{ or } \cong \varprojlim J.
  \]%
\end{definition}
There are two equivalent ways to describe limits in a preorder. First, we can
describe a limit by its membership function
\begin{observation}\label{obs:limit-pointwise}
  Given a diagram \(J : \pwrap{D,\leq} \to \pwrap{X,\leq}\) and an element \(z
  \in X\), \(z\) is a limit for \(J\) if and only if, for any element \(x \in
  X\),
  \[%
    \dset z(x) = \bigwedge_{i \in D} \dset J(i) (x).
  \]%
\end{observation}
\begin{proposition}
  In a preorder \(\pwrap{X,\leq}\), limits are infima. Formally, given a diagram
  \(J : \pwrap{D,\leq} \to \pwrap{X,\leq}\),
  \[%
    \varprojlim J \cong \inf \bwrap{ J(i) \in X \mid i \in D }.
  \]%
\end{proposition}
\begin{proof}
  The universal property of the infimum on the right is: for any element \(x \in
  X\),
  \[%
    x \leq \inf \bwrap{ J(i) \in X \mid i \in D } \text{ iff } \bigwedge_{i \in
      D} x \leq J(i).
  \]%
  By the preceding observation this is the same universal property as the limit
  \(\varprojlim J\).
\end{proof}
\begin{warning_box*}
  The reason that limits are the same as infima for preorders is because the
  classifying category \(\Cl\pwrap{X,\leq}\) of a preorder is \emph{thin}. That
  is, between any two objects \(x,y\) of \(\Cl\pwrap{X,\leq}\), there is at most
  one arrow \(x \to y\). More generally, we need to impose certain commutativity
  conditions which we will explore later.
\end{warning_box*}
There is one special diagram to explore. The empty set \(\emptyset\) has a
(unique) structure of a preordered set. Therefore, we may consider the empty map
\(J_\emptyset : \emptyset \to \pwrap{X,\leq}\) as a diagram, vacuously. We can
ask if \(X\) has any limits for this diagram. A limit for \(J_\emptyset\) is an
element \(z \in X\) such that
\[%
  \dset z = \bigcap_{\_ \in \emptyset} \dset J(\_) = X,
\]%
since empty intersections yield the entire preorder. This can also be seen by
comparing membership functions: \(x \in \dset z\) if and only if, for any \(\_\)
in \(\emptyset\), \(x \in \dset J(\_)\). Since the latter condition is always
vacuously satisfied, \(x \in \dset z\) is always satisfied, so \(\dset z = X\).
\begin{definition}
  A \textbf{terminal object} is a limit of the empty diagram. For a preorder
  \(X\), a terminal object is the same as a maximum for \(X\).
\end{definition}

\subsection*{The reverse situation}

We have related limits to infima. We now describe the category theory for
suprema. These are called \emph{colimits}. Their definition is more
indirect---at least from the downset perspective. We turn to
Observation~\ref{obs:limit-pointwise}, which describes the downset of a limit as
a large conjunction. We can swap the role of \(z\) and \(x\) on the right
side of that equation. This yields a colimit.
\begin{definition}\label{def:colimit}
  Let \(J : \pwrap{D,\leq} \to \pwrap{X,\leq}\) be a diagram. A \textbf{colimit}
  of \(J\) is an object \(z \in X\) such that, for any element \(x \in X\),
  \[%
    \dset x(z) = \bigwedge_{i \in D} \dset J(i) (x).
  \]%
  We write a colimit using similar notation as the limit:
  \[%
    z \cong \varinjlim_{i \in D} J(i) \cong \varinjlim J.
  \]%
\end{definition}
\begin{warning_box*}
  For emphasis, below is the equation characterizing a limit of the diagram
  \(J\), where the difference to the colimit equation highlighted.
  \[%
    \dset{\color{red}z}({\color{red}x}) = \bigwedge_{i \in D} \dset J(i) (x).
  \]%
\end{warning_box*}
\begin{proposition}
  Given a diagram \(J : \pwrap{D,\leq} \to \pwrap{X,\leq}\), \(z\) is a colimit
  for \(J\) if and only if, for any \(x \in X\),
  \[%
    z \leq x \text{ iff } \forall i \in D, J(i) \leq x.
  \]%
  In other words, \(z \cong \sup \bwrap{ J(i) \mid i \in D }\), or in terms of
  downsets
  \[%
    \dset z = \bigvee_{i \in D} \dset J(i).
  \]%
\end{proposition}
Similar to the conjunctions from earlier, we can form disjunctions using a
colimit.
\begin{definition}
  Let \(\pwrap{X,\leq}\) be a preorded set, and let \(x,y\) be a pair of
  elements of \(X\). Let \(\bwrap{a,b}\) denote the preordered set with no
  relation imposed on \(a\) and \(b\). The function
  \(J : \pwrap{a,b} \to \pwrap{X,\leq}\) sending \(a\) to \(x\) and \(b\) to
  \(y\) is order\-/preserving, so it is a diagram. A \textbf{disjunction} for
  \(x\) and \(y\) is a colimit for this diagram. That is, an element \(x \vee
  y\) is a disjunction for \(x\) and \(y\) if and only if, for any element \(v
  \in X\),
  \[%
    x \vee y \leq v \text{ iff } x \leq v \text{ and } y \leq v.
  \]%
\end{definition}
A disjunction's universal property can be expressed diagramtically, like for
conjunctions (or any limit or colimit, for that matter). Here is the diagram for
the disjunction of \(x\) and \(y\).
\[%
  \begin{tikzcd}
    & x \arrow[d] \arrow[ddr,bend left=30] & \\
    y \arrow[r] \arrow[drr,bend right=30] & x \vee y \arrow[dr,dashed,"\exists
    !"] & \\
    & & v
  \end{tikzcd}
\]%
From the diagram we can see an important relationship between limits and
colimits. If we replace the order \(\leq\) on \(X\) with its reverse \(\geq\),
then the ensuing classifying category \(\Cl\pwrap{X,\geq}\) is the opposite
category of \(\Cl\pwrap{X,\leq}\), also called \(\Cl\pwrap{X,\leq}^\op\). All
the arrows in this category have flipped direction. If we flip the arrows in the
above diagram, we get the same picture for a conjunction, just rotated \(180\)
degrees (which does not change the universal property!). In other words:
\begin{observation}
  A disjunction in \(\pwrap{X,\leq}\) is the same as a conjunction in
  \(\pwrap{X,\geq}\).
\end{observation}
More generally, a colimit of \(\Cl\pwrap{X,\leq}\) is the same as a limit of
\(\Cl\pwrap{X,\leq}^\op\). This arrow\-/reversing relationship is the reason
\(x\) and \(z\) are swapped in Definition~\ref{def:colimit}.

We wrap things up by describing the colimit analogue of a terminal object.
\begin{definition}
  Let \(\pwrap{X,\leq}\) be a preordered set. An \textbf{initial object} of
  \(X\) is a colimit for the empty diagram \(J_\emptyset : \emptyset \to
  \pwrap{X,\leq}\).
\end{definition}
Equivalently, an initial object is an element \(z \in X\) such that, for any
\(x \in X\),
\[%
  z \leq x \text{ iff } \forall \_ \in \emptyset, J_\emptyset(\_) \leq x.
\]%
The latter condition is always true, vacuously. Therefore, \(z \leq x\) for
every \(x \in X\), so an initial object of a preorder is the same as a minimum.
\subsection*{Indexed propositions}

The key takeaway from these constructions of limits and colimits is that they
are described by operations on the downsets of elements of
\(\pwrap{X,\leq}\). While \(X\) may not be a Heyting algebra, we can inherit the
language of Heyting algebras by working in \(\Dwd(X)\): the downset of a limit
is an intersection of downsets, and the downset of a colimit is a union of
downsets.

We may take an alternative but equivalent approach. Recall (from the
Grothendieck construction for preorders) that a downward closed set is the same
as a functor \(\Cl\pwrap{X,\leq}^\op \to \Omega\), where \(\Omega\) is the
preorder of truth values. Under this correspondence, intersection of downward
closed sets is identified with the operation
\[%
  \texttt{and} : \Omega \times \Omega \to \Omega,
\]%
and union of downward closed sets is identified with the dual
\[%
  \texttt{or} : \Omega \times \Omega \to \Omega.
\]%
For example, for any \(v \in X\)
\[% 
  \dset \pwrap{x \wedge y}(v) = \dset x (v) \,\texttt{and}\, \dset y (v).
\]%
Indeed, for any operation on propositions \(\Omega \times \ldots \times \Omega
\to \Omega\), we can create a universal property on downsets/indexed
propositions to explore in a preorder.
\begin{definition}
  Given an operation \(\delta : \Omega^n \to \Omega\),
  this operation is \textbf{representable} in a preorder \(\pwrap{X,\leq}\) if,
  for any family of objects \(x_1,\ldots,x_n\), there exists an object \(z \in
  X\) such that for any \(v \in X\)
  \[%
    \dset z (v) = \delta\pwrap{\dset x_1(v),\ldots,\dset x_n(v)}.
  \]%
  Dually, \(\delta\) is \textbf{co\-/representable} in \(\pwrap{X,\leq}\) if it
  is representable in \(\pwrap{X,\geq}\). Equivalently, \(\delta\) is
  corepresentable if, for any family of objects \(x_1,\ldots,x_n\), there exists
  an object \(z \in X\) such that for any \(v \in X\)
  \[%
    \dset v (z) = \delta\pwrap{\dset x_1(v),\ldots,\dset x_n(v)}.
  \]%
\end{definition}
In this sense, a preorder has conjunctions if \(\texttt{and}\) is
representable. This notion of representability will later allow us to describe
the relation between a topos and a universe of sets. In this sense, a topos will
be a category where many of the operations from set theory are either
representable or co\-/representable.

\section{Yoneda and Generalized Objects}

Given a preorder \(\pwrap{X,\leq}\) the elements \(x \in X\) are subsumed by
their downsets \(\dset x \in \Dwd(X)\). We saw this in the Yoneda lemma, where
the preorder \(\leq\) on \(X\) is modeled by the subset inclusion \(\subseteq\)
on \(\Dwd(X)\). We also saw this in the previous section, where universal
properties, such as infima and suprema, can be described in terms of unions and
intersections of downward closed sets. Regarding the other downward-closed sets
\(A \in \Dwd(X)\), since they are downward-closed, they behave like elements in
\(X\), as far as the preordering relation is concerned. For this reason,
downward-closed sets (like their categorified cousins, presheaves) are
\emph{generalized objects} of \(X\). One way to witness this is through the
\emph{co-Yoneda lemma}.
\begin{lemma}[Co-Yoneda Lemma for preorders]
  Let \(A \in \Dwd(X)\) be a downward closed set. The membership function
  \(\chi_A : \Cl\pwrap{X,\leq}^\op \to \Omega\) satisfies the equation
  \[%
    \chi_A(y) = \bigvee_{x \in X} \pwrap{\chi_A(x) \wedge \dset x(y)} =
    \bigvee_{x \in A} \dset x(y).
  \]%
\end{lemma}
The co-Yoneda lemmma shows that any downward-closed set \(A \in \Dwd(X)\) is a
union of sets of the form \(\dset x\), for objects \(x \in X\). These equations
come from the co-Yoneda lemma for presheaves.
\[%
  \begin{array}{ccc}
    \bigvee_{x \in X} \pwrap{\chi_A(x) \wedge \dset x}%
    & \text{ is a special case of }%
    & \int^{x : X} \chi_A(x) \cdot \Yo(x).\\
    \bigvee_{x \in A} \dset x%
    & \text{ is a special case of }%
    & \varinjlim\pwrap{\text{el}(A) \to X \xrightarrow{\Yo} \text{Psh}(X)}.
  \end{array}
\]%
We will present two proofs. The first is a direct proof. The second follows the
structure of the proof of the general Co-Yoneda lemma. Before the proofs, we
discuss how the co-Yoneda lemma allows us to treat downward closed sets like
generalized objects.

\begin{proof}[First proof]
  Let \(y\) be an arbitrary element of \(X\). If \(\chi_A(y) = \True\), then
  \(\chi_A(y) \wedge \dset y (y)\) is also \(\True\). Therefore the right side
  of the equation is \(\True\). Conversely, suppose
  \(\bigvee_{x \in X} \pwrap{\chi_A(x) \wedge \dset x(y)} = \True\). Then there
  exists some \(x_0 \in X\) such that
  \(\chi_A(x_0) \wedge \dset x_0 (y) = \True\). In particular, \(x_0 \in A\) and
  \(y \leq x_0\). Since \(A\) is downward closed, this implies \(y \in A\), so
  \(\chi_A(y) = \True\). Thus, the left side of the equation evaluates to
  \(\True\) if and only if the right side of the equation evaluates to
  \(\True\), as desired.
\end{proof}
\begin{proof}[Second proof]
  We begin by making a few observations. First, we can impose a preorder on
  order\-/reversing functions \(f,g : \pwrap{X,\leq} \to \Omega\) by defining
  \[%
    f \leq g \text{ if and only if } \forall x \in X. f(x) \leq g(x).
  \]%
  Secondly, given two order\-/reversing functions
  \(f_1,f_2 : \pwrap{X,\leq} \to \Omega\), \(f_1 = f_2\) if and only if, for any
  other order\-/reversing function \(g: \pwrap{X,\leq} \to \Omega\), \(f_1 \leq
  g\) if and only if \(f_2 \leq g\).\footnote{%
    The second observation is the Yoneda lemma. We can use the same trick,
    supplying \(f_1\) and \(f_2\) for \(g\), to prove it. Indeed, one could simply
    invoke the Yoneda lemma for preorders, because the class of order\-/preserving
    functions between two preordered sets is a set. When generalizing to presheaves,
    this is no longer the case, so one encounters the usual set-theoretic obstacles.
  }%
  Furthermore, given two order\-/reversing functions \(f,g : \pwrap{X,\leq} \to
  \Omega\), we may define a new function \(f \Rightarrow g : \pwrap{X,\leq} \to
  \Omega\) using the usual implication
  \[%
    \pwrap{f\Rightarrow g}(x \in X) \eqdef f(x) \Rightarrow g(x) =
    \begin{cases}
      \True & f(x) \leq g(x) \\
      \False & \text{otherwise}.
    \end{cases}
  \]%
  Lastly, we can combine the Yoneda lemma for preorders and the last observation
  to deduce \(\pwrap{\dset x \Rightarrow g}(y) = g(x)\) for any input
  \(y \in X\). Based on these observations, in order to prove the desired
  equation
  \[%
    \chi_A(y) = \bigvee_{x \in X} \pwrap{\chi_A(x) \wedge \dset x(y)},
  \]%
  it suffices to prove the following. For any order\-/reversing \(g: X \to
  \Omega\),
  \[%
    \chi_A \leq g \text{ if and only if } \bigvee_{x \in X} \pwrap{\chi_A(x)
      \wedge \dset x} \leq g.
  \]%
  We can now chase a chain of implications
  \begin{align*}
    & \bigvee_{x \in X} \pwrap{\chi_A(x) \wedge \dset x} \leq g \\
    \text{iff} & \bigwedge_{x \in X} \pwrap{ \pwrap{\chi_A(x) \wedge \dset x}
      \leq g }\\
    \text{iff} & \bigwedge_{x \in X} \pwrap{\chi_A(x) \leq \dset x \Rightarrow
      g}\\
    \text{iff} & \bigwedge_{x \in X} \pwrap{\chi_A(x) \leq g(x)}\\
    \text{iff} & \,\chi_A \leq g.
  \end{align*}
\end{proof}

\end{document}
